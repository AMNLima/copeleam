%TCIDATA{LaTeXparent=0,0,main.tex}

\chapter{Figuras e tabelas}

Você também pode incluir gráficos nos formatos \texttt{EPS} ou \texttt{PNG}. A escolha por um destes formatos é feita no preambulo do arquivo \texttt{main.tex}. 

A Figura~\ref{dinosaur} representa sua postura atual frente as mudanças.
Você está ciente do efeito \texttt{orloff} mas não quer pagar para
ver. \begin{figure}[htbp]
\centering \includegraphics[width=80mm]{dinossau} \caption{Voce antes de usar o LaTeX!}%
\label{dinosaur}%
\end{figure}

Já a Figura~\ref{toucan} ilustra como você se sente depois de refletir
um pouco. As pressões são muitas mas não dá para fechar
questão! Note que nao é tao estranho se se sentir assim, vários
personagens ilustres do cenário nacional também se sentem deste modo.
\begin{figure}[htbp]
\centering \includegraphics[width=80mm]{tucanobr} \caption{Voce prestes a usar o LaTeX}%
\label{toucan}%
\end{figure}

\section{Formato EPS}

Para editar as suas figuras sugere-se a utilização de qualquer
programa gráfico que disponibilize um filtro para Encapsulated PostScript
\cite{Doron92e}. Uma vez que sua figura estiver pronta (observe cuidadosamente as dimensões), exporte-a como \texttt{EPS without TIFF Preview}. Lembre de alterar alterar o preambulo, de modo que as opções sejam:

\noindent\verb|\usepackage[dvips]{graphicx}|

\noindent\verb|\graphicspath{{cls/},{src/eps/}}|

\noindent\verb|\DeclareGraphicsExtensions{.eps}|\\
Se você mantiver estas escolhas, armazene seus arquivos gráficos no diretório \texttt{src/eps/}.

\section{Formato PNG}
Para editar as suas figuras sugere-se a utilização de qualquer
programa gráfico que disponibilize um filtro para Portable Network Graphics
\cite{Christensson}. Lembre de alterar alterar o preambulo, de modo que as opções sejam:

\noindent\verb|\usepackage[pdftex]{graphicx}|

\noindent\verb|\graphicspath{{cls/},{src/png/}}|

\noindent\verb|\DeclareGraphicsExtensions{.png}|\\
Se você mantiver estas escolhas, armazene seus arquivos gráficos no diretório \texttt{src/png/}.

\section{Tabelas}

Eis algumas tabelas criadas com comandos do \LaTeX . Eventualmente alguma ou
alguma delas pode lhe ser útil. Dê uma olhada e use-as como exemplos para
construir as suas!

As diversas ferramentas usadas para a preparacao de textos científicos
tem, todas, suas vantagens e desvantagens. A título de exemplo,
apresenta-se na Tabela~\ref{comparando} uma comparação entre algumas
ferramentas de edição de textos.
\begin{table}[ptb]
\centering
\begin{tabular}
[c]{|c|c|c|}\hline
Ferramenta & Curva de aprendizado & Suporte\\\hline
FrameMaker & \multicolumn{1}{|c|}{5.0} & 6.0\\\hline
Troff & \multicolumn{1}{|c|}{10.0} & 1.0\\\hline
\TeX  & \multicolumn{1}{|c|}{7.0} & 10.0\\\hline
Scientific Word/LaTeX & \multicolumn{1}{|c|}{8.0} & 10.0\\\hline
\end{tabular}
\caption{Comparando ferramentas de edição de textos}%
\label{comparando}%
\end{table}
%
\begin{table}[h]
\centering
\caption{Um nome qualquer}
\begin{tabular}{r|lr}
Posi{\c c}{\~a}o & Pa{\'i}s & IDH \\ % Note a separação de col. e a quebra de linhas
\hline                               % para uma linha horizontal
1 & Noruega        & .955 \\
2 & Austr{\'a}lia  & .938 \\
3 & EUA            & .937 \\
4 & Holanda        & .921 \\
5 & Alemanha       & .920            % não é preciso quebrar a última linha

\end{tabular}
\end{table}
%
\begin{table}[htbp]
  \centering
  \caption{Primeiro exemplo de tabela.}
  \label{tab:table1}
  \begin{tabular}{l|c||r}
    1 & 2 & 3\\
    \hline
    a & b & c\\
  \end{tabular}
\end{table}
%
\begin{table}[htbp]
  \centering
  \caption{Segundo exemplo de tabela.}
  \label{tab:table2}
  \begin{tabular}{l|c||r}
    4 & 5 & 6\\
    \hline
    d & e & f\\
  \end{tabular}
\end{table}
%
\begin{table}[htbp]
  \centering
  \caption{Terceiro exemplo de tabela.}
  \label{tab:table3}
  \begin{tabular}{l|c||r}
    1 & 2 & 3\\
    4 & 5 & 6\\
    \hline
    d & e & f\\
    a & b & c\\
  \end{tabular}
\end{table}
%
