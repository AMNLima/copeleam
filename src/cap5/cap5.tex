%TCIDATA{LaTeXparent=0,0,main.tex}

\chapter{Pictures}

Leslie Lamport~\citeauthor{lamport:latex} says "Thinking doesn't guarantee that we won't make mistakes. But not thinking guarantees that we will.". 

\section{\LaTeX and \TeX}
You can create pictures within \LaTeX {}\ using a limited set of
picture symbols. These include vector, line, oval, and others. For
more information you should refer to the \LaTeX {}\ manual
\cite{lamport:latex}.
\begin{figure}[htbp]
\centering  
\begin{picture}(100,100)(0,0)
\put(60,70){\vector(1,0){15}}
\put(65,80){$\theta^+$}
\put(60,40){\line(1,2){20}}
\put(60,40){\line(0,1){40}}
\put(60,40){\oval(40,40)[l]}
\put(60,20){\line(0,1){40}}
\put(0,40){\vector(1,0){40}}
\put(0,43){flow}
\put(20,22){\line(10,0){60}}
\put(20,22){\circle{20}}
\put(60,22){\oval(40,20)[r]}
\end{picture}
\caption{Simple picture created with \LaTeX\ .}
\end{figure}

The above picture was made with the following \LaTeX {}\ code:
\begin{verbatim}
\begin{figure}[htbp]
\centering  
\begin{picture}(100,100)(0,0)
\put(60,70){\vector(1,0){15}}
\put(65,80){$\theta^+$}
\put(60,40){\line(1,2){20}}
\put(60,40){\line(0,1){40}}
\put(60,40){\oval(40,40)[l]}
\put(60,20){\line(0,1){40}}
\put(0,40){\vector(1,0){40}}
\put(0,43){flow}
\put(20,22){\line(10,0){60}}
\put(20,22){\circle{20}}
\put(60,22){\oval(40,20)[r]}
\end{picture}
\caption{Simple picture created with \LaTeX\ .}
\end{figure}
\end{verbatim}
%
\begin{figure}[htbp]
\centering  
\setlength{\unitlength}{1cm} 
\begin{picture}(5,5)(0,0)
\linethickness{2pt} 
\put(0,0){\vector(1,2){1}}	
\put(2,3){\vector(2,3){1}}	
\put(2.2,3.2){$a_1$}	
\end{picture} 
\caption{Simple vectors created with \LaTeX\ .}
\end{figure}
%
\begin{figure}[htbp]
\centering  
\setlength{\unitlength}{1pt} 
\begin{picture}(200,200)(0,0) 
\linethickness{2pt} 
\bezier{20}(0,0)(10,30)(50,30) 
\bezier{200}(0,0)(40,0)(50,30) 
\thinlines 
\put(0,0){\circle*{1}} 
\put(0,0){\line(1,3){10}}	
\put(0,-1){\makebox(0,0)[t]{A}}	
\put(10,30){\circle*{1}} 
\put(10,31){\makebox(0,0)[b]{B}} 
\put(50,30){\circle*{1}} 
\put(50,30){\line(-1,0){40}} 
\put(50,31){\makebox(0,0)[b]{C}} 
\end{picture} 
\caption{Simple arc picture created with \LaTeX\ .}
\end{figure}
%

Another way to get pictures in your \LaTeX \ document is to use
plain \TeX {}\ commands; see The \TeX \ book \cite{knuth:tex}.

\section{TikZ}
There are other and more powerful tools to create graphic elements in LATEX. Tikz is probably the most complex and powerful tool to create graphics for \LaTeX\ . Here follows an example to show the basic features of TikZ package to draw a quite common control system block diagram.
\tikzstyle{block} = [draw, fill=blue!20, rectangle, 
    minimum height=3em, minimum width=6em]
\tikzstyle{sum} = [draw, fill=blue!20, circle, node distance=2cm]
\tikzstyle{input} = [coordinate]
\tikzstyle{output} = [coordinate]
\tikzstyle{pinstyle} = [pin edge={to-,thin,black}]
\begin{figure}[!ht]
\centering
% The block diagram code is probably more verbose than necessary
\begin{tikzpicture}[auto, node distance=3cm,>=latex']
    % We start by placing the blocks
    \node [input, name=input] {};
    \node [sum, right of=input] (sum) {};
    \node [block, right of=sum] (controller) {Controller};
    \node [block, right of=controller, pin={[pinstyle]above:Disturbances},
            node distance=4cm] (system) {System};
    % We draw an edge between the controller and system block to 
    % calculate the coordinate u. We need it to place the measurement block. 
    \draw [->] (controller) -- node[name=u] {$u$} (system);
    \node [output, right of=system] (output) {};
    \node [block, below of=u] (measurements) {Measurements};
    % Once the nodes are placed, connecting them is easy. 
    \draw [draw,->] (input) -- node {$r$} (sum);
    \draw [->] (sum) -- node {$e$} (controller);
    \draw [->] (system) -- node [name=y] {$y$}(output);
    \draw [->] (y) |- (measurements);
    \draw [->] (measurements) -| node[pos=0.99] {$-$} 
        node [near end] {$y_m$} (sum);
\end{tikzpicture}
\caption{Block diagram of a closed-loop control system.}
\end{figure}
The above picture was made with the following TikZ code:

\begin{verbatim}
\tikzstyle{block} = [draw, fill=blue!20, rectangle, 
    minimum height=3em, minimum width=6em]
\tikzstyle{sum} = [draw, fill=blue!20, circle, node distance=2cm]
\tikzstyle{input} = [coordinate]
\tikzstyle{output} = [coordinate]
\tikzstyle{pinstyle} = [pin edge={to-,thin,black}]
\begin{figure}[!ht]
\centering
% The block diagram code is probably more verbose than necessary
\begin{tikzpicture}[auto, node distance=3cm,>=latex']
    % We start by placing the blocks
    \node [input, name=input] {};
    \node [sum, right of=input] (sum) {};
    \node [block, right of=sum] (controller) {Controller};
    \node [block, right of=controller, pin={[pinstyle]above:Disturbances},
            node distance=4cm] (system) {System};
    % We draw an edge between the controller and system block to 
    % calculate the coordinate u. We need it to place the measurement block. 
    \draw [->] (controller) -- node[name=u] {$u$} (system);
    \node [output, right of=system] (output) {};
    \node [block, below of=u] (measurements) {Measurements};
    % Once the nodes are placed, connecting them is easy. 
    \draw [draw,->] (input) -- node {$r$} (sum);
    \draw [->] (sum) -- node {$e$} (controller);
    \draw [->] (system) -- node [name=y] {$y$}(output);
    \draw [->] (y) |- (measurements);
    \draw [->] (measurements) -| node[pos=0.99] {$-$} 
        node [near end] {$y_m$} (sum);
\end{tikzpicture}
\caption{Block diagram of a closed-loop control system.}
\end{figure}
\end{verbatim}
