Exponha suas motivações profissionais e pessoais
Nenhuma pesquisa surge do nada. Mesmo que seja algo pouco abordado ou ainda não reconhecido em sua área, você precisa mostrar que está motivado a ser o primeiro a tratar do assunto ou a fazer releituras que melhorem temáticas já trabalhadas.

É na dissertação de mestrado que o estudante trará algo novo, que poderá até mesmo se encaminhar para uma tese de doutorado. Então, esteja motivado e demonstre toda a sua empolgação na introdução.

Cative o seu leitor: O leitor é nossa maior preocupação na hora de escrever um texto. É preciso considerar que não será só a banca que terá o prazer de ler o seu trabalho, pois ele também servirá de base para outras pesquisas e, até mesmo, para você continuar evoluindo na área.

Portanto, considere todos os seus possíveis leitores. Uma escrita clara, sem muitos rodeios e que realmente apresente o que a sua dissertação propõe é uma boa maneira de conquistar o leitor e tornar a sua introdução uma ótima apresentação do trabalho.

Esclareça as ideias apresentadas na introdução: Todas as ideias apresentadas na introdução precisam ser esclarecidas quanto à forma como serão tratadas no decorrer das páginas. Ainda na seção introdutória, suas colocações já precisam responder aos objetivos propostos para a pesquisa.

A clareza na escrita deixa a leitura mais leve. Quando você mostra na introdução que suas ideias são objetivas, o leitor terá prazer em seguir para o primeiro capítulo.

Revise o seu texto: Como já dissemos, algumas pessoas gostam de escrever a introdução antes do restante do texto. Isso é possível, porém não se deve desconsiderar a necessidade de uma revisão mais sistemática nesses casos, pois o percurso durante a pesquisa — que dura, em média, de um a dois anos — pode mudar (e muito!).

Mesmo quando escrevemos depois do trabalho já pronto, a revisão é essencial para garantir a coesão e a coerência da introdução em relação a todo o texto. Uma introdução mal escrita pode custar a falta de interesse do leitor, que poderá ignorar a sua pesquisa e partir para outra.

Organize a introdução da sua dissertação de mestrado
Agora que estamos preparados para trazer, na introdução, aspectos relevantes para conquistar o leitor, vamos falar um pouco mais sobre os principais pontos estruturais que essa seção deve conter:

Levante uma problemática geral: Toda introdução deve apresentar as partes de um projeto de pesquisa — geralmente, escrito antes do início das pesquisas, como proposta de trabalho. Nessa parte, você deverá apresentar a temática, a área de concentração e o problema levantado para ser pesquisado.

Tenha objetivos claros: A partir do problema, você precisa identificar seus objetivos e trazer bons argumentos que justifiquem a importância de sua pesquisa. Explicite o objetivo geral da sua dissertação e, também, os objetivos específicos, que são como uma ramificação da questão maior.

Justifique sua pesquisa: Para justificar seu trabalho, você poderá trazer autores que já tratem do assunto de forma renomada ou questões que mostrem a necessidade de sua pesquisa para a evolução da área de estudo e até mesmo da humanidade, quando couber.

Apresente a estrutura do seu trabalho: Por fim, apresente as metodologias usadas durante sua pesquisa, além das principais referências. Mostre como elas se concretizaram em sua dissertação de mestrado.

Você pode ir descrevendo um pouco de cada capítulo, situando o leitor sobre cada parte de seu trabalho. Mas não fale coisas desnecessárias: para deixá-lo com vontade de ler o seu texto, coloque pontos relevantes.

Algo que faz com que a introdução de uma dissertação de mestrado seja considerada ótima é a clareza demonstrada na linguagem que você utiliza.
