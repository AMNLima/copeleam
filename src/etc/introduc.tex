%TCIDATA{LaTeXparent=0,0,main.tex}

Este roteiro foi preparado na expectativa de ajudar os autores através dos
exemplos apresentados ao longo deste documento. A idéia básica é
que os autores copiem os arquivos relativos a este roteiro e façam as
modificações que acharem convenientes para produzirem seus documentos
finais. Se ao copiar os arquivos exemplos você manteve os mesmos nomes
você deve estar editando o arquivo \texttt{main.tex}. Nas
instruções e comentários que seguem admite-se que o arquivo
principal chama-se \texttt{main.tex} e que os arquivos secundários
nele referenciados, direta ou indiretamente, são:

\noindent\texttt{introduc.tex}, \texttt{orientad.tex}, \texttt{membrosb.tex},
\texttt{dedicato.tex}, \texttt{agradece.tex},

\noindent\texttt{resumopt.tex}, \texttt{resumoen.tex}
\texttt{cap2.tex}, \texttt{cap3.tex}, \texttt{cap4.tex},

\noindent\texttt{cap5.tex}, \texttt{apendice.tex},
\texttt{conclusa.tex} e \texttt{main.bib}.

Sugiro que voce altere somente os nomes do arquivo principal e do arquivo de
referências e mantenha os demais nomes que já são bastante
sugestivos e mnemônicos. Se você alterou estes nomes, é sua
responsabilidade adaptar o restante das referências do roteiro-exemplo
para compatibilizá-las com a escolha que voce fez. Seguindo este roteiro,
você deve iniciar a edição neste documento, modificando os campos

\noindent\texttt{Titulo}, \texttt{Autor}, \texttt{Data} e
\texttt{Area}

Em princípio, iste é tudo que você precisa mudar neste documento.
As outras modificações, você deve faze-las, nos arquivos
secundários referenciados neste arquivo.

Depois de alterar estes campos você também deve editar o conteúdo dos seguintes arquivos:

\noindent\texttt{orientad.tex}, \texttt{membrosb.tex}, \texttt{dedicato.tex},
\texttt{agradece.tex},

\noindent\texttt{resumopt.tex} e \texttt{resumoen.tex}.

Edite estes arquivos, sem modificar seus nomes, e altere seus conteudos onde necessario.

Teses e Dissertações são trabalhos escritos, individuais e
originais. Uma tese de doutoramento é um trabalho científico original
que apresenta uma reflexão aprofundada sobre um tema específico,
nunca antes tratado e cujo resultado final constitui uma contribuição
valiosa e única para o conhecimento do tema tratado. Não esqueça
que uma \texttt{tese} é uma proposição que se expõe para ser
defendida; este é o sentido original de uma tese escrita para
obtenção do grau de doutor. O grau de doutor comprova a
realização de uma contribuição inovadora e original para o
progresso do conhecimento, um alto nível cultural numa determinada
área do conhecimento e a aptidão para realizar trabalho
científico independente. O grau de mestre comprova nível aprofundado
de conhecimentos numa área científica específica e capacidade
para a prática da investigação. As diferenças são, pois,
de natureza formal e de conteúdo: espera-se de uma dissertação de
mestrado que seja um trabalho mais breve do que uma tese de doutoramento.

\section{Método científico}
A execução do seu trabalho de pesquisa, independente do seu nível deve ser pautada pelo que convenciona denominar de método científico. 
\citeauthor{knuth:tex} says "Science is what we understand well enough to explain to a computer. Art is everything else we do". De forma resumida o método científico enseja um algoritmo composto das seguintes etapas:

\begin{itemize}
\item ETAPA 1: Descobrimento do problema ou lacuna num conjunto de conhecimentos.  Caso o problema não esteja enunciado com clareza e precisão, passa-se à Etapa 2; se estiver, segue-se à Etapa 3.
\item ETAPA 2: Colocação precisa do problema. Aqui, o problema deve ser recolocado à luz de novos conhecimentos articulados ou no processo de articulação.
\item ETAPA 3: Procura de conhecimentos ou instrumentos relevantes ao problema.  Nesta etapa, o pesquisador deverá levar em consideração as teorias, os dados empíricos, as tecnologias existentes, para, a partir do conhecido, tentar resolver o problema. 
\item ETAPA 4: Tentativa de solução do problema com o auxílio dos meios identificados.  Caso esta tentativa não logre êxito, deve-se passar para a Etapa 5; do contrário, passa-se à Etapa 6;
\item ETAPA 5: Invenção de novas ideias (hipóteses, teorias ou técnicas) ou produção de novos dados empíricos que possibilitem uma solução razoável ao problema;
\item ETAPA 6: Obtenção de uma solução próxima ou exata para o problema a partir dos instrumentos conceituais ou empíricos disponíveis;
\item ETAPA 7: Investigação das consequências da solução obtida. No caso de uma teoria, devem-se procurar os prognósticos que possam ser feitos com o seu auxílio. Se forem novos dados, devem-se examinar as consequências que possam ter para as teorias existentes e relevantes;
\item ETAPA 8: Prova (comprovação) da solução. Aqui, a solução encontrada deve ser confrontada com a totalidade das teorias e das informações empíricas pertinentes. Caso o resultado seja satisfatório, a pesquisa pode ser dada por concluída até que novos problemas surjam. No contrário, deve-se passar para a Etapa 9.
\item ETAPA 9: Correção das hipóteses, teorias, procedimentos ou dados empregados na obtenção da solução incorreta, passa-se à Etapa 2. Caso isto venha a ocorrer, estar-se diante do começo de um novo ciclo de investigação, caminho natural de qualquer indivíduo que queira buscar novos conhecimentos.
\end{itemize}

\section{Mensagem introdutória}
Este documento é um roteiro para auxiliar os autores de
dissertações e teses na edição destes trabalhos.
Este roteiro é especificamente destinado aos autores que optarem
por editar seus trabalhos usando o \TeX\ \cite{knuth:tex,texbook}\abreviatura{TeX}{Formatting engine\nomrefpage} ou o \LaTeX\ \cite{lamport:latex,latexbook}\abreviatura{LaTeX}{Generic typesetting system that uses TeX as its formatting engine\nomrefpage}.
Este roteiro foi preparado na expectativa de ajudar os autores
através dos exemplos apresentados ao longo deste documento. A
idéia básica é que os autores copiem os arquivos
relativos a este roteiro e façam as modificações que
acharem convenientes para gerarem seus documentos finais.

