%TCIDATA{LaTeXparent=0,0,main.tex}

Este roteiro foi preparado na expectativa de ajudar os autores através dos
exemplos apresentados ao longo deste documento. A idéia básica é
que os autores copiem os arquivos relativos a este roteiro e façam as
modificações que acharem convenientes para produzirem seus documentos
finais. Se ao copiar os arquivos exemplos você manteve os mesmos nomes
você deve estar editando o arquivo \texttt{main.tex}. Nas
instruções e comentários que seguem admite-se que o arquivo
principal chama-se \texttt{main.tex} e que os arquivos secundários
nele referenciados, direta ou indiretamente, são:

\noindent\texttt{introduc.tex}, \texttt{orientad.tex}, \texttt{membrosb.tex},
\texttt{dedicato.tex}, \texttt{agradece.tex},

\noindent\texttt{resumopt.tex}, \texttt{resumoen.tex}
\texttt{cap2.tex}, \texttt{cap3.tex}, \texttt{cap4.tex},

\noindent\texttt{cap5.tex}, \texttt{apendice.tex},
\texttt{conclusa.tex} e \texttt{main.bib}.

Sugiro que voce altere somente os nomes do arquivo principal e do arquivo de
referências e mantenha os demais nomes que já são bastante
sugestivos e mnemônicos. Se você alterou estes nomes, é sua
responsabilidade adaptar o restante das referências do roteiro-exemplo
para compatibilizá-las com a escolha que voce fez. Seguindo este roteiro,
você deve iniciar a edição neste documento, modificando os campos

\noindent\texttt{Titulo}, \texttt{Autor}, \texttt{Data} e
\texttt{Area}

Em princípio, iste é tudo que você precisa mudar neste documento.
As outras modificações, você deve faze-las, nos arquivos
secundários referenciados neste arquivo.

Depois de alterar estes campos você também deve editar o conteúdo dos seguintes arquivos:

\noindent\texttt{orientad.tex}, \texttt{membrosb.tex}, \texttt{dedicato.tex},
\texttt{agradece.tex},

\noindent\texttt{resumopt.tex} e \texttt{resumoen.tex}.

Edite estes arquivos, sem modificar seus nomes, e altere seus conteudos onde necessario.

Depois disto, altere o conteúdo deste arquivo e inclua aqui a introdução do seu trabalho. Lembre que Teses e Dissertações são trabalhos escritos, individuais e
originais. Uma tese de doutoramento é um trabalho científico original
que apresenta uma reflexão aprofundada sobre um tema específico,
nunca antes tratado e cujo resultado final constitui uma contribuição
valiosa e única para o conhecimento do tema tratado. Não esqueça
que uma \texttt{tese} é uma proposição que se expõe para ser
defendida; este é o sentido original de uma tese escrita para
obtenção do grau de doutor. O grau de doutor comprova a
realização de uma contribuição inovadora e original para o
progresso do conhecimento, um alto nível cultural numa determinada
área do conhecimento e a aptidão para realizar trabalho
científico independente. Por outro lado, o grau de mestre comprova nível aprofundado
de conhecimentos numa área científica específica e capacidade
para a prática da investigação. As diferenças são, pois,
de natureza formal e de conteúdo: espera-se de uma dissertação de
mestrado que seja um trabalho mais simples e breve do que uma tese de doutoramento.

Na introdução você deve contextualizar e apresentar o problema que pretende resolver. Na apresentação do problema é necessário tratar da relevância científica e tecnológica do problema, bem como da área na qual ele se insere. Além da relevância é importante tratar a atualidade e da complexidade do problema. 

\section{Método científico}
A execução do seu trabalho de pesquisa, independente do seu nível deve ser pautada pelo que convenciona denominar de método científico. 
\citeauthor{knuth:tex} says "Science is what we understand well enough to explain to a computer. Art is everything else we do". De forma resumida o método científico enseja um algoritmo composto das seguintes etapas:

\begin{enumerate}
\item \label{etapa1} Descobrimento do problema ou lacuna num conjunto de conhecimentos.  Caso o problema não esteja enunciado com clareza e precisão, passa-se à Etapa~\ref{etapa2}; se estiver, segue-se à Etapa~\ref{etapa3}.
\item \label{etapa2} Colocação precisa do problema. Aqui, o problema deve ser recolocado à luz de novos conhecimentos articulados ou no processo de articulação.
\item \label{etapa3} Procura de conhecimentos ou instrumentos relevantes ao problema.  Nesta etapa, o pesquisador deverá levar em consideração as teorias, os dados empíricos, as tecnologias existentes, para, a partir do conhecido, tentar resolver o problema. 
\item \label{etapa4} Tentativa de solução do problema com o auxílio dos meios identificados.  Caso esta tentativa não logre êxito, deve-se passar para a Etapa~\ref{etapa5}; do contrário, passa-se à Etapa~\ref{etapa6};
\item \label{etapa5} Invenção de novas ideias (hipóteses, teorias ou técnicas) ou produção de novos dados empíricos que possibilitem uma solução razoável ao problema;
\item \label{etapa6} Obtenção de uma solução próxima ou exata para o problema a partir dos instrumentos conceituais ou empíricos disponíveis;
\item \label{etapa7} Investigação das consequências da solução obtida. No caso de uma teoria, devem-se procurar os prognósticos que possam ser feitos com o seu auxílio. Se forem novos dados, devem-se examinar as consequências que possam ter para as teorias existentes e relevantes;
\item \label{etapa8} Prova (comprovação) da solução. Aqui, a solução encontrada deve ser confrontada com a totalidade das teorias e das informações empíricas pertinentes. Caso o resultado seja satisfatório, a pesquisa pode ser dada por concluída até que novos problemas surjam. No contrário, deve-se passar para a Etapa~\ref{etapa9}.
\item \label{etapa9} Correção das hipóteses, teorias, procedimentos ou dados empregados na obtenção da solução incorreta, passa-se à Etapa~\ref{etapa2}. Caso isto venha a ocorrer, estar-se diante do começo de um novo ciclo de investigação, caminho natural de qualquer indivíduo que queira buscar novos conhecimentos.
\end{enumerate}

\section{Mensagem introdutória}
Este documento é um roteiro para auxiliar os autores de
dissertações e teses na edição destes trabalhos.
Este roteiro é especificamente destinado aos autores que optarem
por editar seus trabalhos usando o \TeX\ \cite{knuth:tex,texbook}\abreviatura{TeX}{Formatting engine\nomrefpage} ou o \LaTeX\ \cite{lamport:latex,latexbook}\abreviatura{LaTeX}{Generic typesetting system that uses TeX as its formatting engine\nomrefpage}.
Este roteiro foi preparado na expectativa de ajudar os autores
através dos exemplos apresentados ao longo deste documento. A
idéia básica é que os autores copiem os arquivos
relativos a este roteiro e façam as modificações que
acharem convenientes para gerarem seus documentos finais.

\section{Sua introdução}
Como você deve escrever a introdução do seu trabalho? Considere os seguintes tópicos antes de escrever a introdução do seu projeto de pós-graduação:

\subsection{Contexto}
Contextualize o tema do seu projeto e explique em que consiste o problema que você irá resolver, suas especificidades e as razões que justicam o objeto de sua pesquisa. Use argumentos técnicos e cientificos consistentes para explicar como o seu trabalho é relevante e  atual. Uma boa contextualização é essencial para legitimar seu projeto de pesquisa.

\subsection{Inovação}
É necessário avaliar os trabalhos dos outros autores que propuseram soluções para o mesmo problema que você decidiu estudar. Responder à pergunta: Quem já estudou o mesmo problema? Quais as soluções já propostas? Se há mais de uma solução, quais são as respectivas vantagens e desvantagens? Nessa parte do texto, deve-se justificar e explicar o "que será feito no trabalho"\ e não o "como será feito o trabalho". O "como será feito o trabalho"\ deve ser portegardo para a parte do trabalho que sucede a "revisão bibliográfica", a qual é denominada de "metodologia". Na redação do objetivo, 

\subsection{Objetivo}
Definido o contexto, a relevância, a atualidade do objeto de estudo, e as limitações das soluções anteriormente propostas, deve-se apresentar o objetivo do trabalho em tela. De modo geral, o objetivo é dividido em "objetivo geral"\ e "objetivos específicos". Na seção denominada de "objetivo geral"\ deve-se apresentar a ideia central de um trabalho acadêmico. Na seção denominada de "objetivos específicos"\ deve-se detalhar as etapas necessárias para a consecução do "objetivo geral".

Exponha suas motivações profissionais e pessoais
Nenhuma pesquisa surge do nada. Mesmo que seja algo pouco abordado ou ainda não reconhecido em sua área, você precisa mostrar que está motivado a ser o primeiro a tratar do assunto ou a fazer releituras que melhorem temáticas já trabalhadas.

É na dissertação de mestrado que o estudante trará algo novo, que poderá até mesmo se encaminhar para uma tese de doutorado. Então, esteja motivado e demonstre toda a sua empolgação na introdução.

Cative o seu leitor: O leitor é nossa maior preocupação na hora de escrever um texto. É preciso considerar que não será só a banca que terá o prazer de ler o seu trabalho, pois ele também servirá de base para outras pesquisas e, até mesmo, para você continuar evoluindo na área.

Portanto, considere todos os seus possíveis leitores. Uma escrita clara, sem muitos rodeios e que realmente apresente o que a sua dissertação propõe é uma boa maneira de conquistar o leitor e tornar a sua introdução uma ótima apresentação do trabalho.

Esclareça as ideias apresentadas na introdução: Todas as ideias apresentadas na introdução precisam ser esclarecidas quanto à forma como serão tratadas no decorrer das páginas. Ainda na seção introdutória, suas colocações já precisam responder aos objetivos propostos para a pesquisa.

A clareza na escrita deixa a leitura mais leve. Quando você mostra na introdução que suas ideias são objetivas, o leitor terá prazer em seguir para o primeiro capítulo.

Revise o seu texto: Como já dissemos, algumas pessoas gostam de escrever a introdução antes do restante do texto. Isso é possível, porém não se deve desconsiderar a necessidade de uma revisão mais sistemática nesses casos, pois o percurso durante a pesquisa — que dura, em média, de um a dois anos — pode mudar (e muito!).

Mesmo quando escrevemos depois do trabalho já pronto, a revisão é essencial para garantir a coesão e a coerência da introdução em relação a todo o texto. Uma introdução mal escrita pode custar a falta de interesse do leitor, que poderá ignorar a sua pesquisa e partir para outra.

Organize a introdução da sua dissertação de mestrado
Agora que estamos preparados para trazer, na introdução, aspectos relevantes para conquistar o leitor, vamos falar um pouco mais sobre os principais pontos estruturais que essa seção deve conter:

Levante uma problemática geral: Toda introdução deve apresentar as partes de um projeto de pesquisa — geralmente, escrito antes do início das pesquisas, como proposta de trabalho. Nessa parte, você deverá apresentar a temática, a área de concentração e o problema levantado para ser pesquisado.

Tenha objetivos claros: A partir do problema, você precisa identificar seus objetivos e trazer bons argumentos que justifiquem a importância de sua pesquisa. Explicite o objetivo geral da sua dissertação e, também, os objetivos específicos, que são como uma ramificação da questão maior.

Justifique sua pesquisa: Para justificar seu trabalho, você poderá trazer autores que já tratem do assunto de forma renomada ou questões que mostrem a necessidade de sua pesquisa para a evolução da área de estudo e até mesmo da humanidade, quando couber.

Apresente a estrutura do seu trabalho: Por fim, apresente as metodologias usadas durante sua pesquisa, além das principais referências. Mostre como elas se concretizaram em sua dissertação de mestrado.

Você pode ir descrevendo um pouco de cada capítulo, situando o leitor sobre cada parte de seu trabalho. Mas não fale coisas desnecessárias: para deixá-lo com vontade de ler o seu texto, coloque pontos relevantes.

Algo que faz com que a introdução de uma dissertação de mestrado seja considerada ótima é a clareza demonstrada na linguagem que você utiliza.

\section{Organização do trabalho}
No fim do capítulo de introdução você deve explicar como está organizado o texto, indicando quais os assuntos serão estudados nos respectivos capítulos, destacando suas eventuais dependências. No demais capítulos essa seção também deve existir, porém denominada de Considerações finais na qual declaram-se as conclusões do capítulo e comenta-se os vínculos com os demais capítulos subsequentes.
