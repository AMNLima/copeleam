%TCIDATA{LaTeXparent=0,0,main.tex}

O seu documento deve ter um capítulo de conclusão. Neste capítulo você deve tratar dos aspectos-chave do seu trabalho e também das limitações do estudo. Você deve referenciar os resultados que você obteve en termos dos objetivos definidos na introdução do trabalho. Você deve enfatizar a importância do estudo, tanto no mundo acadêmico quanto não-acadêmico. 

No fim deste capítulo é importante apresentar sugestões de trabalhos futuros. Entretanto, é bom lembrar que tais sugestões devem emergir da investigação e dos resultados obtidos no seu trabalho.

No caso desse documento uma conclusão esperada é que voce tenha lido este texto e aprendido, pelos exemplos, como preparar seu documento. Lembre que caso o seu documento seja uma dissertação ou uma tese, será disponibilizado para consulta pública pela CAPES. 

Considerando que as informações aqui fornecidas sobre edição de documentos usando \TeX \ e \LaTeX \ são muito resumidas, recomenda-se que você consulte livros especializados em \TeX \ e \LaTeX \ tais como o livro do \TeX \ escrito por Donald E. Knuth \cite{texbook} ou o livro do \LaTeX \ escrito por Leslie Lamport \cite{latexbook}, para suas futuras dúvidas.
\abreviatura{CAPES}{Coordenação de Aperfeiçoamento de Pessoal de N\'{i}vel Superior\nomrefpage}
\abreviatura{CNPq}{Conselho Nacional de Desenvolvimento Científico e Tecnológico\nomrefpage}%
\abreviatura{FINEP}{Financiadora de Estudos e Projetos\nomrefpage}%
\abreviatura{BNDES}{Banco Nacional de Desenvolvimento Econômico e Social\nomrefpage}%
\abreviatura{FIESP}{Federação das Indústrias do Estado de São Paulo\nomrefpage}%
