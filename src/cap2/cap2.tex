%TCIDATA{LaTeXparent=0,0,main.tex}

\chapter{Seguindo o exemplo\label{cap2}}

Dissertações e teses, invariavelmente, iniciam pela
apresentação do tema sobre o qual versam \cite{SP}. Esta
apresentação também inclui a motivação, a
contextualização do tema e a revisao bibliográfica. A
apresentação, a motivação, a contextualização e a
revisão bibliográfica constituem a introdução de seu trabalho.
De modo geral, no final da introdução descreve-se a
organização do trabalho.

Elabore a introdução de seu trabalho, gere num
arquivo chamado \texttt{introduc.tex} e armazene-o, obrigatoriamente, no diretório \texttt{src/etc/}; este arquivo será incluído automaticamente pelo comando \textbackslash\texttt{Introducao}. De forma semelhante, elabore a conclusão de seu trabalho, gere um arquivo chamado \texttt{conclusa.tex} e armazene-o, obrigatoriamente, no diretório \texttt{src/etc/}; este arquivo será incluído automaticamente pelo comando \textbackslash\texttt{Conclusao}. 

Se voce edita estes arquivos separadamente, e.g., usando o \texttt{Scientific Word}, lembre-se que você deve eliminar, com ajuda de um editor \texttt{ASCII},
e.g., \texttt{notepad}, \texttt{textedit}, \texttt{qe}, \texttt{vi} ou \texttt{edit}, todos os comandos
\begin{verbatim}
\documentclass[...]{...} ... \begin{document}
\end{verbatim}
\noindent do preâmbulo do documento e, também o comando
\begin{verbatim}
\end{document}
\end{verbatim}

Entretanto, se você preferir usar os arquivos disponibilizados como exemplos isto não será necessário.

\section{Símbolos e abreviaturas}

Os trabalhos de dissertação de mestrado e tese de doutorado contém
siglas, fórrmulas e equaões, cujos significados devem ser
inequívocos. A definição das siglas, dos parâmetros e
variáveis que voce usa ao longo do seu trabalho deve ser apresentada logo
no início do documento. Defina seus símbolos e abreviaturas e
insira, onde julgar apropriado, os comandos

\noindent\verb|\simbolo{símbolo}{definição\nomrefpage}|,

\noindent\verb|\abreviatura{abreviatura}{definição\nomrefpage}|,

\noindent adaptados do pacote \texttt{nomencl}. Se você utilizar estes comandos a sua \texttt{Lista de símbolos e abreviaturas} será gerada automaticamente; veja, o exemplo em seguida, no qual há a definição de símbolos para vetor de campo elétrico, vetor de campo magnético, velocidade da luz, permissividade elétrica do espaço livre e permeabilidade magnética do espaço livre, que vão ser incluídos na Lista de Símbolos e abreviaturas.

As equações de Maxwell permitem descrever de forma concisa e elegante
os fundamentos da eletricidade e magnetismo. A formulação integral das
equações de Maxwell para o caso em que o meio não é polarizavél e não há ímãs é:%
\begin{align}
\oint\overrightarrow{\mathbf{E}}\cdot d\overrightarrow{\mathbf{A}}  &
=\frac{q}{\varepsilon_{0}}\\
\oint\overrightarrow{\mathbf{B}}\cdot d\overrightarrow{\mathbf{A}}  & =0\\
\oint\overrightarrow{\mathbf{E}}\cdot d\overrightarrow{\mathbf{s}}  &
=-\frac{d\Phi_{\mathbf{B}}}{dt}\\
\oint\overrightarrow{\mathbf{B}}\cdot d\overrightarrow{\mathbf{s}}  & =\mu
_{0}i+\frac{1}{c^{2}}\frac{\partial}{\partial t}\oint\overrightarrow
{\mathbf{E}}\cdot d\overrightarrow{\mathbf{A}}
\end{align}

\simbolo{$\overrightarrow{\mathbf{E}}$}{Vetor de campo elétrico, Vm$^{-1}$\nomrefpage}%
\simbolo{$i_{k}\left(t\right)$}{Valor instantaneo da corrente elétrica que flui no $k$-ésimo ramo do circuito, A\nomrefpage}%
\simbolo{$v_{l}\left(t\right)$}{Valor instantaneo da tensão elétrica do $l$-ésimo nó, A\nomrefpage}%
\simbolo{$\overrightarrow{\mathbf{B}}$}{Vetor de campo magnético, T\nomrefpage}%
\simbolo{$c$}{Velocidade da luz, ms$^{-1}$\nomrefpage}%
\simbolo{$\varepsilon_{0}$}{Permissividade elétrica do espaço livre, C$^{2}$N$^{-1}$m$^{-2}$\nomrefpage}%
\simbolo{$\mu_{0}$}{Permeabilidade magnética do espaço livre, NA$^{-2}$\nomrefpage}%
\simbolo{$T\left(s\right)$}{Função de Sensibilidade Complementar\nomrefpage}%
\simbolo{$S\left(s\right)$}{Função de Sensibilidade\nomrefpage}%
\simbolo{$L\left(s\right)$}{Ganho de malha\nomrefpage}%
\simbolo{$K$}{Ganho da lei de realimentação de estados\nomrefpage}%
\simbolo{$\lambda_{i}$}{$i$-ésimo autovalor da matriz $\mathbf{M}$\nomrefpage}%
\simbolo{$p_{i}$}{$i$-ésimo pólo da função de transferência $H\left(s\right)$\nomrefpage}%
\simbolo{$z_{j}$}{$j$-ésimo zero da função de transferência $H\left(s\right)$\nomrefpage}%

\footnotetext{O arquivo \texttt{introduc.tex}, se existir no diret\'{o}rio \texttt{src/etc/}, será incluído automaticamente pelo \LaTeX \ no momento da compilação, através do comando \textbackslash\texttt{Introducao}. Você pode, se preferir, preparar este arquivo na etapa final da redação do seu documento.}

Este é o primeiro capítulo depois da introdução e será numerado, no documento impresso, como Capítulo~\ref{cap2} \cite{taylor}; a introdução é sempre o primeiro capítulo. É aqui onde começa, efetivamente, o texto relativo ao desenvolvimento de seu trabalho \cite{etex}. Você pode armazenar este texto num arquivo único ou dividi-lo em
vários arquivos separados. O texto deste roteiro, por exemplo,
está dividido da seguinte maneira:

\noindent\texttt{src/etc/introduc.tex}, \texttt{src/cap2/cap2.tex},
\texttt{src/cap3/cap3.tex}, \texttt{src/cap4/cap4.tex} e \texttt{src/etc/conclusa.tex}

Estes arquivos são arquivos secundários associados ao arquivo
principal \cite{AbTaRu:54}. Usando a terminologia do \texttt{Scientific Word},
estes arquivos são subdocumentos de \texttt{main.tex}. O arquivo
\begin{verbatim}
src/cap2/cap2.tex
\end{verbatim}
refere-se ao documento que voce está editando
\cite{IEEEexample:masterstype}.

\section{Incluindo arquivos}

Veja que no arquivo original, \texttt{main.tex}, as instruções de
inclusão que referem-se aos seguintes arquivos secundários:

\noindent\verb|%TCIDATA{LaTeXparent=0,0,main.tex}

\chapter{Seguindo o exemplo\label{cap2}}

Dissertações e teses, invariavelmente, iniciam pela
apresentação do tema sobre o qual versam \cite{SP}. Esta
apresentação também inclui a motivação, a
contextualização do tema e a revisao bibliográfica. A
apresentação, a motivação, a contextualização e a
revisão bibliográfica constituem a introdução de seu trabalho.
De modo geral, no final da introdução descreve-se a
organização do trabalho.

Elabore a introdução de seu trabalho, gere num
arquivo chamado \texttt{introduc.tex} e armazene-o, obrigatoriamente, no diretório \texttt{src/etc/}; este arquivo será incluído automaticamente pelo comando \textbackslash\texttt{Introducao}. De forma semelhante, elabore a conclusão de seu trabalho, gere um arquivo chamado \texttt{conclusa.tex} e armazene-o, obrigatoriamente, no diretório \texttt{src/etc/}; este arquivo será incluído automaticamente pelo comando \textbackslash\texttt{Conclusao}. 

Se voce edita estes arquivos separadamente, e.g., usando o \texttt{Scientific Word}, lembre-se que você deve eliminar, com ajuda de um editor \texttt{ASCII},
e.g., \texttt{notepad}, \texttt{textedit}, \texttt{qe}, \texttt{vi} ou \texttt{edit}, todos os comandos
\begin{verbatim}
\documentclass[...]{...} ... \begin{document}
\end{verbatim}
\noindent do preâmbulo do documento e, também o comando
\begin{verbatim}
\end{document}
\end{verbatim}

Entretanto, se você preferir usar os arquivos disponibilizados como exemplos isto não será necessário.

\section{Símbolos e abreviaturas}

Os trabalhos de dissertação de mestrado e tese de doutorado contém
siglas, fórrmulas e equaões, cujos significados devem ser
inequívocos. A definição das siglas, dos parâmetros e
variáveis que voce usa ao longo do seu trabalho deve ser apresentada logo
no início do documento. Defina seus símbolos e abreviaturas e
insira, onde julgar apropriado, os comandos

\noindent\verb|\simbolo{símbolo}{definição\nomrefpage}|,

\noindent\verb|\abreviatura{abreviatura}{definição\nomrefpage}|,

\noindent adaptados do pacote \texttt{nomencl}. Se você utilizar estes comandos a sua \texttt{Lista de símbolos e abreviaturas} será gerada automaticamente; veja, o exemplo em seguida, no qual há a definição de símbolos para vetor de campo elétrico, vetor de campo magnético, velocidade da luz, permissividade elétrica do espaço livre e permeabilidade magnética do espaço livre, que vão ser incluídos na Lista de Símbolos e abreviaturas.

As equações de Maxwell permitem descrever de forma concisa e elegante
os fundamentos da eletricidade e magnetismo. A formulação integral das
equações de Maxwell para o caso em que o meio não é polarizavél e não há ímãs é:%
\begin{align}
\oint\overrightarrow{\mathbf{E}}\cdot d\overrightarrow{\mathbf{A}}  &
=\frac{q}{\varepsilon_{0}}\\
\oint\overrightarrow{\mathbf{B}}\cdot d\overrightarrow{\mathbf{A}}  & =0\\
\oint\overrightarrow{\mathbf{E}}\cdot d\overrightarrow{\mathbf{s}}  &
=-\frac{d\Phi_{\mathbf{B}}}{dt}\\
\oint\overrightarrow{\mathbf{B}}\cdot d\overrightarrow{\mathbf{s}}  & =\mu
_{0}i+\frac{1}{c^{2}}\frac{\partial}{\partial t}\oint\overrightarrow
{\mathbf{E}}\cdot d\overrightarrow{\mathbf{A}}
\end{align}

\simbolo{$\overrightarrow{\mathbf{E}}$}{Vetor de campo elétrico, Vm$^{-1}$\nomrefpage}%
\simbolo{$i_{k}\left(t\right)$}{Valor instantaneo da corrente elétrica que flui no $k$-ésimo ramo do circuito, A\nomrefpage}%
\simbolo{$v_{l}\left(t\right)$}{Valor instantaneo da tensão elétrica do $l$-ésimo nó, A\nomrefpage}%
\simbolo{$\overrightarrow{\mathbf{B}}$}{Vetor de campo magnético, T\nomrefpage}%
\simbolo{$c$}{Velocidade da luz, ms$^{-1}$\nomrefpage}%
\simbolo{$\varepsilon_{0}$}{Permissividade elétrica do espaço livre, C$^{2}$N$^{-1}$m$^{-2}$\nomrefpage}%
\simbolo{$\mu_{0}$}{Permeabilidade magnética do espaço livre, NA$^{-2}$\nomrefpage}%
\simbolo{$T\left(s\right)$}{Função de Sensibilidade Complementar\nomrefpage}%
\simbolo{$S\left(s\right)$}{Função de Sensibilidade\nomrefpage}%
\simbolo{$L\left(s\right)$}{Ganho de malha\nomrefpage}%
\simbolo{$K$}{Ganho da lei de realimentação de estados\nomrefpage}%
\simbolo{$\lambda_{i}$}{$i$-ésimo autovalor da matriz $\mathbf{M}$\nomrefpage}%
\simbolo{$p_{i}$}{$i$-ésimo pólo da função de transferência $H\left(s\right)$\nomrefpage}%
\simbolo{$z_{j}$}{$j$-ésimo zero da função de transferência $H\left(s\right)$\nomrefpage}%

\footnotetext{O arquivo \texttt{introduc.tex}, se existir no diret\'{o}rio \texttt{src/etc/}, será incluído automaticamente pelo \LaTeX \ no momento da compilação, através do comando \textbackslash\texttt{Introducao}. Você pode, se preferir, preparar este arquivo na etapa final da redação do seu documento.}

Este é o primeiro capítulo depois da introdução e será numerado, no documento impresso, como Capítulo~\ref{cap2} \cite{taylor}; a introdução é sempre o primeiro capítulo. É aqui onde começa, efetivamente, o texto relativo ao desenvolvimento de seu trabalho \cite{etex}. Você pode armazenar este texto num arquivo único ou dividi-lo em
vários arquivos separados. O texto deste roteiro, por exemplo,
está dividido da seguinte maneira:

\noindent\texttt{src/etc/introduc.tex}, \texttt{src/cap2/cap2.tex},
\texttt{src/cap3/cap3.tex}, \texttt{src/cap4/cap4.tex} e \texttt{src/etc/conclusa.tex}

Estes arquivos são arquivos secundários associados ao arquivo
principal \cite{AbTaRu:54}. Usando a terminologia do \texttt{Scientific Word},
estes arquivos são subdocumentos de \texttt{main.tex}. O arquivo
\begin{verbatim}
src/cap2/cap2.tex
\end{verbatim}
refere-se ao documento que voce está editando
\cite{IEEEexample:masterstype}.

\section{Incluindo arquivos}

Veja que no arquivo original, \texttt{main.tex}, as instruções de
inclusão que referem-se aos seguintes arquivos secundários:

\noindent\verb|%TCIDATA{LaTeXparent=0,0,main.tex}

\chapter{Seguindo o exemplo\label{cap2}}

Dissertações e teses, invariavelmente, iniciam pela
apresentação do tema sobre o qual versam \cite{SP}. Esta
apresentação também inclui a motivação, a
contextualização do tema e a revisao bibliográfica. A
apresentação, a motivação, a contextualização e a
revisão bibliográfica constituem a introdução de seu trabalho.
De modo geral, no final da introdução descreve-se a
organização do trabalho.

Elabore a introdução de seu trabalho, gere num
arquivo chamado \texttt{introduc.tex} e armazene-o, obrigatoriamente, no diretório \texttt{src/etc/}; este arquivo será incluído automaticamente pelo comando \textbackslash\texttt{Introducao}. De forma semelhante, elabore a conclusão de seu trabalho, gere um arquivo chamado \texttt{conclusa.tex} e armazene-o, obrigatoriamente, no diretório \texttt{src/etc/}; este arquivo será incluído automaticamente pelo comando \textbackslash\texttt{Conclusao}. 

Se voce edita estes arquivos separadamente, e.g., usando o \texttt{Scientific Word}, lembre-se que você deve eliminar, com ajuda de um editor \texttt{ASCII},
e.g., \texttt{notepad}, \texttt{textedit}, \texttt{qe}, \texttt{vi} ou \texttt{edit}, todos os comandos
\begin{verbatim}
\documentclass[...]{...} ... \begin{document}
\end{verbatim}
\noindent do preâmbulo do documento e, também o comando
\begin{verbatim}
\end{document}
\end{verbatim}

Entretanto, se você preferir usar os arquivos disponibilizados como exemplos isto não será necessário.

\section{Símbolos e abreviaturas}

Os trabalhos de dissertação de mestrado e tese de doutorado contém
siglas, fórrmulas e equaões, cujos significados devem ser
inequívocos. A definição das siglas, dos parâmetros e
variáveis que voce usa ao longo do seu trabalho deve ser apresentada logo
no início do documento. Defina seus símbolos e abreviaturas e
insira, onde julgar apropriado, os comandos

\noindent\verb|\simbolo{símbolo}{definição\nomrefpage}|,

\noindent\verb|\abreviatura{abreviatura}{definição\nomrefpage}|,

\noindent adaptados do pacote \texttt{nomencl}. Se você utilizar estes comandos a sua \texttt{Lista de símbolos e abreviaturas} será gerada automaticamente; veja, o exemplo em seguida, no qual há a definição de símbolos para vetor de campo elétrico, vetor de campo magnético, velocidade da luz, permissividade elétrica do espaço livre e permeabilidade magnética do espaço livre, que vão ser incluídos na Lista de Símbolos e abreviaturas.

As equações de Maxwell permitem descrever de forma concisa e elegante
os fundamentos da eletricidade e magnetismo. A formulação integral das
equações de Maxwell para o caso em que o meio não é polarizavél e não há ímãs é:%
\begin{align}
\oint\overrightarrow{\mathbf{E}}\cdot d\overrightarrow{\mathbf{A}}  &
=\frac{q}{\varepsilon_{0}}\\
\oint\overrightarrow{\mathbf{B}}\cdot d\overrightarrow{\mathbf{A}}  & =0\\
\oint\overrightarrow{\mathbf{E}}\cdot d\overrightarrow{\mathbf{s}}  &
=-\frac{d\Phi_{\mathbf{B}}}{dt}\\
\oint\overrightarrow{\mathbf{B}}\cdot d\overrightarrow{\mathbf{s}}  & =\mu
_{0}i+\frac{1}{c^{2}}\frac{\partial}{\partial t}\oint\overrightarrow
{\mathbf{E}}\cdot d\overrightarrow{\mathbf{A}}
\end{align}

\simbolo{$\overrightarrow{\mathbf{E}}$}{Vetor de campo elétrico, Vm$^{-1}$\nomrefpage}%
\simbolo{$i_{k}\left(t\right)$}{Valor instantaneo da corrente elétrica que flui no $k$-ésimo ramo do circuito, A\nomrefpage}%
\simbolo{$v_{l}\left(t\right)$}{Valor instantaneo da tensão elétrica do $l$-ésimo nó, A\nomrefpage}%
\simbolo{$\overrightarrow{\mathbf{B}}$}{Vetor de campo magnético, T\nomrefpage}%
\simbolo{$c$}{Velocidade da luz, ms$^{-1}$\nomrefpage}%
\simbolo{$\varepsilon_{0}$}{Permissividade elétrica do espaço livre, C$^{2}$N$^{-1}$m$^{-2}$\nomrefpage}%
\simbolo{$\mu_{0}$}{Permeabilidade magnética do espaço livre, NA$^{-2}$\nomrefpage}%
\simbolo{$T\left(s\right)$}{Função de Sensibilidade Complementar\nomrefpage}%
\simbolo{$S\left(s\right)$}{Função de Sensibilidade\nomrefpage}%
\simbolo{$L\left(s\right)$}{Ganho de malha\nomrefpage}%
\simbolo{$K$}{Ganho da lei de realimentação de estados\nomrefpage}%
\simbolo{$\lambda_{i}$}{$i$-ésimo autovalor da matriz $\mathbf{M}$\nomrefpage}%
\simbolo{$p_{i}$}{$i$-ésimo pólo da função de transferência $H\left(s\right)$\nomrefpage}%
\simbolo{$z_{j}$}{$j$-ésimo zero da função de transferência $H\left(s\right)$\nomrefpage}%

\footnotetext{O arquivo \texttt{introduc.tex}, se existir no diret\'{o}rio \texttt{src/etc/}, será incluído automaticamente pelo \LaTeX \ no momento da compilação, através do comando \textbackslash\texttt{Introducao}. Você pode, se preferir, preparar este arquivo na etapa final da redação do seu documento.}

Este é o primeiro capítulo depois da introdução e será numerado, no documento impresso, como Capítulo~\ref{cap2} \cite{taylor}; a introdução é sempre o primeiro capítulo. É aqui onde começa, efetivamente, o texto relativo ao desenvolvimento de seu trabalho \cite{etex}. Você pode armazenar este texto num arquivo único ou dividi-lo em
vários arquivos separados. O texto deste roteiro, por exemplo,
está dividido da seguinte maneira:

\noindent\texttt{src/etc/introduc.tex}, \texttt{src/cap2/cap2.tex},
\texttt{src/cap3/cap3.tex}, \texttt{src/cap4/cap4.tex} e \texttt{src/etc/conclusa.tex}

Estes arquivos são arquivos secundários associados ao arquivo
principal \cite{AbTaRu:54}. Usando a terminologia do \texttt{Scientific Word},
estes arquivos são subdocumentos de \texttt{main.tex}. O arquivo
\begin{verbatim}
src/cap2/cap2.tex
\end{verbatim}
refere-se ao documento que voce está editando
\cite{IEEEexample:masterstype}.

\section{Incluindo arquivos}

Veja que no arquivo original, \texttt{main.tex}, as instruções de
inclusão que referem-se aos seguintes arquivos secundários:

\noindent\verb|%TCIDATA{LaTeXparent=0,0,main.tex}

\chapter{Seguindo o exemplo\label{cap2}}

Dissertações e teses, invariavelmente, iniciam pela
apresentação do tema sobre o qual versam \cite{SP}. Esta
apresentação também inclui a motivação, a
contextualização do tema e a revisao bibliográfica. A
apresentação, a motivação, a contextualização e a
revisão bibliográfica constituem a introdução de seu trabalho.
De modo geral, no final da introdução descreve-se a
organização do trabalho.

Elabore a introdução de seu trabalho, gere num
arquivo chamado \texttt{introduc.tex} e armazene-o, obrigatoriamente, no diretório \texttt{src/etc/}; este arquivo será incluído automaticamente pelo comando \textbackslash\texttt{Introducao}. De forma semelhante, elabore a conclusão de seu trabalho, gere um arquivo chamado \texttt{conclusa.tex} e armazene-o, obrigatoriamente, no diretório \texttt{src/etc/}; este arquivo será incluído automaticamente pelo comando \textbackslash\texttt{Conclusao}. 

Se voce edita estes arquivos separadamente, e.g., usando o \texttt{Scientific Word}, lembre-se que você deve eliminar, com ajuda de um editor \texttt{ASCII},
e.g., \texttt{notepad}, \texttt{textedit}, \texttt{qe}, \texttt{vi} ou \texttt{edit}, todos os comandos
\begin{verbatim}
\documentclass[...]{...} ... \begin{document}
\end{verbatim}
\noindent do preâmbulo do documento e, também o comando
\begin{verbatim}
\end{document}
\end{verbatim}

Entretanto, se você preferir usar os arquivos disponibilizados como exemplos isto não será necessário.

\section{Símbolos e abreviaturas}

Os trabalhos de dissertação de mestrado e tese de doutorado contém
siglas, fórrmulas e equaões, cujos significados devem ser
inequívocos. A definição das siglas, dos parâmetros e
variáveis que voce usa ao longo do seu trabalho deve ser apresentada logo
no início do documento. Defina seus símbolos e abreviaturas e
insira, onde julgar apropriado, os comandos

\noindent\verb|\simbolo{símbolo}{definição\nomrefpage}|,

\noindent\verb|\abreviatura{abreviatura}{definição\nomrefpage}|,

\noindent adaptados do pacote \texttt{nomencl}. Se você utilizar estes comandos a sua \texttt{Lista de símbolos e abreviaturas} será gerada automaticamente; veja, o exemplo em seguida, no qual há a definição de símbolos para vetor de campo elétrico, vetor de campo magnético, velocidade da luz, permissividade elétrica do espaço livre e permeabilidade magnética do espaço livre, que vão ser incluídos na Lista de Símbolos e abreviaturas.

As equações de Maxwell permitem descrever de forma concisa e elegante
os fundamentos da eletricidade e magnetismo. A formulação integral das
equações de Maxwell para o caso em que o meio não é polarizavél e não há ímãs é:%
\begin{align}
\oint\overrightarrow{\mathbf{E}}\cdot d\overrightarrow{\mathbf{A}}  &
=\frac{q}{\varepsilon_{0}}\\
\oint\overrightarrow{\mathbf{B}}\cdot d\overrightarrow{\mathbf{A}}  & =0\\
\oint\overrightarrow{\mathbf{E}}\cdot d\overrightarrow{\mathbf{s}}  &
=-\frac{d\Phi_{\mathbf{B}}}{dt}\\
\oint\overrightarrow{\mathbf{B}}\cdot d\overrightarrow{\mathbf{s}}  & =\mu
_{0}i+\frac{1}{c^{2}}\frac{\partial}{\partial t}\oint\overrightarrow
{\mathbf{E}}\cdot d\overrightarrow{\mathbf{A}}
\end{align}

\simbolo{$\overrightarrow{\mathbf{E}}$}{Vetor de campo elétrico, Vm$^{-1}$\nomrefpage}%
\simbolo{$i_{k}\left(t\right)$}{Valor instantaneo da corrente elétrica que flui no $k$-ésimo ramo do circuito, A\nomrefpage}%
\simbolo{$v_{l}\left(t\right)$}{Valor instantaneo da tensão elétrica do $l$-ésimo nó, A\nomrefpage}%
\simbolo{$\overrightarrow{\mathbf{B}}$}{Vetor de campo magnético, T\nomrefpage}%
\simbolo{$c$}{Velocidade da luz, ms$^{-1}$\nomrefpage}%
\simbolo{$\varepsilon_{0}$}{Permissividade elétrica do espaço livre, C$^{2}$N$^{-1}$m$^{-2}$\nomrefpage}%
\simbolo{$\mu_{0}$}{Permeabilidade magnética do espaço livre, NA$^{-2}$\nomrefpage}%
\simbolo{$T\left(s\right)$}{Função de Sensibilidade Complementar\nomrefpage}%
\simbolo{$S\left(s\right)$}{Função de Sensibilidade\nomrefpage}%
\simbolo{$L\left(s\right)$}{Ganho de malha\nomrefpage}%
\simbolo{$K$}{Ganho da lei de realimentação de estados\nomrefpage}%
\simbolo{$\lambda_{i}$}{$i$-ésimo autovalor da matriz $\mathbf{M}$\nomrefpage}%
\simbolo{$p_{i}$}{$i$-ésimo pólo da função de transferência $H\left(s\right)$\nomrefpage}%
\simbolo{$z_{j}$}{$j$-ésimo zero da função de transferência $H\left(s\right)$\nomrefpage}%

\footnotetext{O arquivo \texttt{introduc.tex}, se existir no diret\'{o}rio \texttt{src/etc/}, será incluído automaticamente pelo \LaTeX \ no momento da compilação, através do comando \textbackslash\texttt{Introducao}. Você pode, se preferir, preparar este arquivo na etapa final da redação do seu documento.}

Este é o primeiro capítulo depois da introdução e será numerado, no documento impresso, como Capítulo~\ref{cap2} \cite{taylor}; a introdução é sempre o primeiro capítulo. É aqui onde começa, efetivamente, o texto relativo ao desenvolvimento de seu trabalho \cite{etex}. Você pode armazenar este texto num arquivo único ou dividi-lo em
vários arquivos separados. O texto deste roteiro, por exemplo,
está dividido da seguinte maneira:

\noindent\texttt{src/etc/introduc.tex}, \texttt{src/cap2/cap2.tex},
\texttt{src/cap3/cap3.tex}, \texttt{src/cap4/cap4.tex} e \texttt{src/etc/conclusa.tex}

Estes arquivos são arquivos secundários associados ao arquivo
principal \cite{AbTaRu:54}. Usando a terminologia do \texttt{Scientific Word},
estes arquivos são subdocumentos de \texttt{main.tex}. O arquivo
\begin{verbatim}
src/cap2/cap2.tex
\end{verbatim}
refere-se ao documento que voce está editando
\cite{IEEEexample:masterstype}.

\section{Incluindo arquivos}

Veja que no arquivo original, \texttt{main.tex}, as instruções de
inclusão que referem-se aos seguintes arquivos secundários:

\noindent\verb|\input{src/cap2/cap2.tex}|

\noindent\verb|\input{src/cap3/cap3.tex}|

\noindent\verb|\input{src/cap4/cap4.tex}|

Se voce prefir mudar os nomes e incluir outros arquivos, voce deve
corrigir as instruções respectivas no arquivo principal
\cite{IEEEexample:phdurl}. Por exemplo, se voce criou os arquivos
\texttt{src/cap5/cap5.tex} e \texttt{src/apendice/apendice.tex}. Voce deve incluir no arquivo principal as seguintes instruções:

\noindent\verb|\input{src/cap5/cap5.tex}|

\noindent\verb|\input{src/apendice/apendice.tex}|

Lembre que estes arquivos são do tipo secundário e não devem ter o cabeçalho padrão do \LaTeX  e nem a instrução de fim de documento, ou sejam,
\begin{verbatim}
\documentclass ... \begin{document} ... \end{document}
\end{verbatim}

Este cabeçalho so é necessário no arquivo \texttt{main.tex}. Por outro
lado, instruções do tipo
\begin{verbatim}
\chapter{nomedocapitulo} ... \section{nomedasecao}
\end{verbatim}
podem e devem ser usadas.

Observe que não é necessário acrescentar no arquivo \texttt{main.tex}
os comandos para incluir os arquivos: \texttt{src/etc/orientad.tex}, \texttt{src/etc/membrosb.tex},
\texttt{src/etc/dedicato.tex}, \texttt{src/etc/agradece.tex}, \texttt{src/etc/resumopt.tex} e \texttt{src/etc/resumoen.tex}.

Estes arquivos, devem ser, obrigatoriamente, armazenados no diretório \texttt{src/etc} e serão incluídos automaticamente no momento da compilação, como é feito no caso dos arquivos \texttt{src/etc/conclusa.tex}, \texttt{src/etc/introduc.tex}.

\section{Recomendações de redação}
Um trabalho científico pode e deve ser consultado por várias pessoas, inclusive de áreas do conhecimento diferentes daquela que você atua. Desse modo, deve-se adotar uma forma expositiva e o nível de clareza interna, fornecendo ao leitor todas as informações necessárias para a compreensão do texto, inclusive a definição dos termos empregados.

Entre algumas sugestões para a estrutura interna do texto, pode-se citar:

– Evitar períodos longos;

– Abrir parágrafos para “arejar” o texto;

– Repetir o sujeito da frase sempre que for necessário para que a compressão não seja prejudicada;

– Eliminar o excesso de pronomes e orações subordinadas;

– Suprimir as divagações, transformando-as em notas ou apêndices;

– Concentrar-se na demonstração das hipóteses levantadas;

– Verificar a facilidade de leitura do texto, solicitando que outras pessoas também o leiam;

– Evitar o emprego de reticências e pontos de exclamação;

– Usar figuras de linguagem apenas quando presumir que o leitor será capaz de compreendê-las;

– Definir um termo técnico ao induzi-lo pela primeira vez no texto;

– Ser coerente na identificação de autores e outras informações referentes às fontes documentais.

Entre as expressões que devem ser evitadas nos textos acadêmicos, sugere-se substituir:

\begin{tabular}{|L{7cm}|L{7cm}|}\hline
Expressão  &	Opções\\\hline
– a nível de, ao nível	& – em nível de, no nível\\\hline
– face a, frente a	&– ante, diante de, em vista de\\\hline
– onde (quando não exprime lugar)	& – em que, na qual, nas quais, no qual\\\hline
– sob um ponto de vista	& – de um ponto de vista\\\hline
– sob um prisma	& – por um prisma\\\hline
– como sendo	& – suprimir a expressão\\\hline
– em função de	& – em virtude de, por causa de, por,\\\hline
– a partir de (sem valor temporal)	& – com base em, tomando-se por base\\\hline
– através de (sem atravessar algo)	& – por meio de, segundo, por, mediante\\\hline
– devido a	& – em razão de, em virtude de\\\hline
– dito	& – citado, mencionado\\\hline
– como um todo	& – total, integral, completo\\\hline
– há anos atrás	& – há anos\\\hline
– por cada	& – suprimir a cacofonia\\\hline
– antes do estudo ser feito	& – antes de o estudo ser realizado\\\hline
– pelo fato destes resultados apontarem o contrário	& – pelo fato de estes resultados apontarem o contrário.\\\hline
\end{tabular}
|

\noindent\verb|%TCIDATA{LaTeXparent=0,0,main.tex}

\chapter{Citações e referências}

De modo geral é necessário fazer referência a trabalhos anteriores
que serviram de base para o desenvolvimento de seu trabalho. As
citações e a apresentação das referências no final do
documento deve seguir o padrão ABNT que pode ser implementado com o pacote

\texttt{abntex2cite}.

\noindent A lista dos trabalhos que são referenciados neste
exemplo é armazenada no arquivo \newline\texttt{src/bib/main.bib}.

\section{Referenciando}

Veja os exemplos: O programa \TeX {}\ foi idealizado e implementado
inicialmente na decada de 70 \cite{knuth:tex}. \LaTeX {}\ \cite{lamport:latex}
é uma das várias extensões do \TeX \ \cite{knuth:tex}. Há
várias implementações de \LaTeX {}\ \cite{lamport:latex} para PCs
\cite{furuta:pctex}.

\section{BibTeX}

Lembre-se de executar o BibTeX\ \cite{patashnik:bibtex} e, em seguida, executar
o \LaTeX \ \cite{lamport:latex} duas vezes consecutivas, sempre que novas
citações forem incluídas no texto.

\section{Referências}
 \lipsum[1-3]|

\noindent\verb|%TCIDATA{LaTeXparent=0,0,main.tex}

\chapter{Figuras e tabelas}

Você também pode incluir gráficos nos formatos \texttt{EPS} ou \texttt{PNG}. A escolha por um destes formatos é feita no preambulo do arquivo \texttt{main.tex}. 

A Figura~\ref{dinosaur} representa sua postura atual frente as mudanças.
Você está ciente do efeito \texttt{orloff} mas não quer pagar para
ver. \begin{figure}[htbp]
\centering \includegraphics[width=80mm]{dinossau} \caption{Voce antes de usar o LaTeX!}%
\label{dinosaur}%
\end{figure}
\begin{figure}[htbp]
\centering \includegraphics[width=80mm]{dinossau} \caption{Voce antes de usar o LaTeX!}%
\label{dinosaur1}%
\end{figure}
\begin{figure}[htbp]
\centering \includegraphics[width=80mm]{dinossau} \caption{Voce antes de usar o LaTeX!}%
\label{dinosaur2}%
\end{figure}
\begin{figure}[htbp]
\centering \includegraphics[width=80mm]{dinossau} \caption{Voce antes de usar o LaTeX!}%
\label{dinosaur3}%
\end{figure}
\begin{figure}[htbp]
\centering \includegraphics[width=80mm]{dinossau} \caption{Voce antes de usar o LaTeX!}%
\label{dinosaur4}%
\end{figure}
\begin{figure}[htbp]
\centering \includegraphics[width=80mm]{dinossau} \caption{Voce antes de usar o LaTeX!}%
\label{dinosaur5}%
\end{figure}
\begin{figure}[htbp]
\centering \includegraphics[width=80mm]{dinossau} \caption{Voce antes de usar o LaTeX!}%
\label{dinosaur6}%
\end{figure}
\begin{figure}[htbp]
\centering \includegraphics[width=80mm]{dinossau} \caption{Voce antes de usar o LaTeX!}%
\label{dinosaur7}%
\end{figure}
\begin{figure}[htbp]
\centering \includegraphics[width=80mm]{dinossau} \caption{Voce antes de usar o LaTeX!}%
\label{dinosaur8}%
\end{figure}
\begin{figure}[htbp]
\centering \includegraphics[width=80mm]{dinossau} \caption{Voce antes de usar o LaTeX!}%
\label{dinosaur9}%
\end{figure}

Já a Figura~\ref{toucan} ilustra como você se sente depois de refletir
um pouco. As pressões são muitas mas não dá para fechar
questão! Note que nao é tao estranho se se sentir assim, vários
personagens ilustres do cenário nacional também se sentem deste modo.
\begin{figure}[htbp]
\centering \includegraphics[width=80mm]{tucanobr} \caption{Voce prestes a usar o LaTeX}%
\label{toucan}%
\end{figure}
\begin{figure}[htbp]
\centering \includegraphics[width=80mm]{tucanobr} \caption{Voce prestes a usar o LaTeX}%
\label{toucan1}%
\end{figure}
\begin{figure}[htbp]
\centering \includegraphics[width=80mm]{tucanobr} \caption{Voce prestes a usar o LaTeX}%
\label{toucan2}%
\end{figure}
\begin{figure}[htbp]
\centering \includegraphics[width=80mm]{tucanobr} \caption{Voce prestes a usar o LaTeX}%
\label{toucan3}%
\end{figure}
\begin{figure}[htbp]
\centering \includegraphics[width=80mm]{tucanobr} \caption{Voce prestes a usar o LaTeX}%
\label{toucan4}%
\end{figure}
\begin{figure}[htbp]
\centering \includegraphics[width=80mm]{tucanobr} \caption{Voce prestes a usar o LaTeX}%
\label{toucan5}%
\end{figure}
\begin{figure}[htbp]
\centering \includegraphics[width=80mm]{tucanobr} \caption{Voce prestes a usar o LaTeX}%
\label{toucan6}%
\end{figure}
\begin{figure}[htbp]
\centering \includegraphics[width=80mm]{tucanobr} \caption{Voce prestes a usar o LaTeX}%
\label{toucan7}%
\end{figure}
\begin{figure}[htbp]
\centering \includegraphics[width=80mm]{tucanobr} \caption{Voce prestes a usar o LaTeX}%
\label{toucan8}%
\end{figure}
\begin{figure}[htbp]
\centering \includegraphics[width=80mm]{tucanobr} \caption{Voce prestes a usar o LaTeX}%
\label{toucan9}%
\end{figure}

\section{Formato EPS}

Para editar as suas figuras sugere-se a utilização de qualquer
programa gráfico que disponibilize um filtro para Encapsulated PostScript
\cite{Doron92e}. Uma vez que sua figura estiver pronta (observe cuidadosamente as dimensões), exporte-a como \texttt{EPS without TIFF Preview}. Lembre de alterar alterar o preambulo, de modo que as opções sejam:

\noindent\verb|\usepackage[dvips]{graphicx}|

\noindent\verb|\graphicspath{{cls/},{src/eps/}}|

\noindent\verb|\DeclareGraphicsExtensions{.eps}|\\
Se você mantiver estas escolhas, armazene seus arquivos gráficos no diretório \texttt{src/eps/}.

\section{Formato PNG}
Para editar as suas figuras sugere-se a utilização de qualquer
programa gráfico que disponibilize um filtro para Portable Network Graphics
\cite{Christensson}. Lembre de alterar alterar o preambulo, de modo que as opções sejam:

\noindent\verb|\usepackage[pdftex]{graphicx}|

\noindent\verb|\graphicspath{{cls/},{src/png/}}|

\noindent\verb|\DeclareGraphicsExtensions{.png}|\\
Se você mantiver estas escolhas, armazene seus arquivos gráficos no diretório \texttt{src/png/}.

\section{Tabelas}

Eis algumas tabelas criadas com comandos do \LaTeX . Eventualmente alguma ou
alguma delas pode lhe ser útil. Dê uma olhada e use-as como exemplos para
construir as suas!

As diversas ferramentas usadas para a preparacao de textos científicos
tem, todas, suas vantagens e desvantagens. A título de exemplo,
apresenta-se na Tabela~\ref{comparando} uma comparação entre algumas
ferramentas de edição de textos.
\begin{table}[ptb]
\centering
\begin{tabular}
[c]{|c|c|c|}\hline
Ferramenta & Curva de aprendizado & Suporte\\\hline
FrameMaker & \multicolumn{1}{|c|}{5.0} & 6.0\\\hline
Troff & \multicolumn{1}{|c|}{10.0} & 1.0\\\hline
\TeX  & \multicolumn{1}{|c|}{7.0} & 10.0\\\hline
Scientific Word/LaTeX & \multicolumn{1}{|c|}{8.0} & 10.0\\\hline
\end{tabular}
\caption{Comparando ferramentas de edição de textos}%
\label{comparando}%
\end{table}
%
\begin{table}[htbp]
\centering
\begin{tabular}
[c]{|c|c|c|}\hline
Ferramenta & Curva de aprendizado & Suporte\\\hline
FrameMaker & \multicolumn{1}{|c|}{5.0} & 6.0\\\hline
Troff & \multicolumn{1}{|c|}{10.0} & 1.0\\\hline
\TeX  & \multicolumn{1}{|c|}{7.0} & 10.0\\\hline
Scientific Word/LaTeX & \multicolumn{1}{|c|}{8.0} & 10.0\\\hline
\end{tabular}
\caption{Comparando ferramentas de edição de textos}%
\label{comparando1}%
\end{table}
%
\begin{table}[h]
\centering
\caption{Um nome qualquer}
\begin{tabular}{r|lr}
Posi{\c c}{\~a}o & Pa{\'i}s & IDH \\ % Note a separação de col. e a quebra de linhas
\hline                               % para uma linha horizontal
1 & Noruega        & .955 \\
2 & Austr{\'a}lia  & .938 \\
3 & EUA            & .937 \\
4 & Holanda        & .921 \\
5 & Alemanha       & .920            % não é preciso quebrar a última linha

\end{tabular}
\end{table}
%
\begin{table}[htbp]
  \centering
  \caption{Caption for the table.}
  \label{tab:table1}
  \begin{tabular}{l|c||r}
    1 & 2 & 3\\
    \hline
    a & b & c\\
  \end{tabular}
\end{table}
%
\begin{table}[htbp]
  \centering
  \caption{Caption for the table.}
  \label{tab:table2}
  \begin{tabular}{l|c||r}
    4 & 5 & 6\\
    \hline
    d & e & f\\
  \end{tabular}
\end{table}
%
\begin{table}[htbp]
  \centering
  \caption{Caption for the table.}
  \label{tab:table3}
  \begin{tabular}{l|c||r}
    1 & 2 & 3\\
    4 & 5 & 6\\
    \hline
    d & e & f\\
    a & b & c\\
  \end{tabular}
\end{table}
%
\begin{table}[htbp]
  \centering
  \caption{Caption for the table.}
  \label{tab:table11}
  \begin{tabular}{l|c||r}
    1 & 2 & 3\\
    \hline
    a & b & c\\
  \end{tabular}
\end{table}
%
\begin{table}[htbp]
  \centering
  \caption{Caption for the table.}
  \label{tab:table12}
  \begin{tabular}{l|c||r}
    4 & 5 & 6\\
    \hline
    d & e & f\\
  \end{tabular}
\end{table}
%
\begin{table}[htbp]
  \centering
  \caption{Caption for the table.}
  \label{tab:table13}
  \begin{tabular}{l|c||r}
    1 & 2 & 3\\
    4 & 5 & 6\\
    \hline
    d & e & f\\
    a & b & c\\
  \end{tabular}
\end{table}
%
\begin{table}[htbp]
  \centering
  \caption{Caption for the table.}
  \label{tab:table21}
  \begin{tabular}{l|c||r}
    1 & 2 & 3\\
    \hline
    a & b & c\\
  \end{tabular}
\end{table}
%
\begin{table}[htbp]
  \centering
  \caption{Caption for the table.}
  \label{tab:table22}
  \begin{tabular}{l|c||r}
    4 & 5 & 6\\
    \hline
    d & e & f\\
  \end{tabular}
\end{table}
%
\begin{table}[htbp]
  \centering
  \caption{Caption for the table.}
  \label{tab:table23}
  \begin{tabular}{l|c||r}
    1 & 2 & 3\\
    4 & 5 & 6\\
    \hline
    d & e & f\\
    a & b & c\\
  \end{tabular}
\end{table}
%
\begin{table}[htbp]
  \centering
  \caption{Caption for the table.}
  \label{tab:table31}
  \begin{tabular}{l|c||r}
    1 & 2 & 3\\
    \hline
    a & b & c\\
  \end{tabular}
\end{table}
%
\begin{table}[htbp]
  \centering
  \caption{Caption for the table.}
  \label{tab:table32}
  \begin{tabular}{l|c||r}
    4 & 5 & 6\\
    \hline
    d & e & f\\
  \end{tabular}
\end{table}
%
\begin{table}[htbp]
  \centering
  \caption{Caption for the table.}
  \label{tab:table33}
  \begin{tabular}{l|c||r}
    1 & 2 & 3\\
    4 & 5 & 6\\
    \hline
    d & e & f\\
    a & b & c\\
  \end{tabular}
\end{table}
%
\begin{table}[htbp]
  \centering
  \caption{Caption for the table.}
  \label{tab:table131}
  \begin{tabular}{l|c||r}
    1 & 2 & 3\\
    \hline
    a & b & c\\
  \end{tabular}
\end{table}
%
\begin{table}[htbp]
  \centering
  \caption{Caption for the table.}
  \label{tab:table132}
  \begin{tabular}{l|c||r}
    4 & 5 & 6\\
    \hline
    d & e & f\\
  \end{tabular}
\end{table}
%
\begin{table}[htbp]
  \centering
  \caption{Caption for the table.}
  \label{tab:table133}
  \begin{tabular}{l|c||r}
    1 & 2 & 3\\
    4 & 5 & 6\\
    \hline
    d & e & f\\
    a & b & c\\
  \end{tabular}
\end{table}
%
\begin{table}[htbp]
  \centering
  \caption{Caption for the table.}
  \label{tab:table231}
  \begin{tabular}{l|c||r}
    1 & 2 & 3\\
    \hline
    a & b & c\\
  \end{tabular}
\end{table}
%
\begin{table}[htbp]
  \centering
  \caption{Caption for the table.}
  \label{tab:table232}
  \begin{tabular}{l|c||r}
    4 & 5 & 6\\
    \hline
    d & e & f\\
  \end{tabular}
\end{table}
%
\begin{table}[htbp]
  \centering
  \caption{Caption for the table.}
  \label{tab:table233}
  \begin{tabular}{l|c||r}
    1 & 2 & 3\\
    4 & 5 & 6\\
    \hline
    d & e & f\\
    a & b & c\\
  \end{tabular}
\end{table}
%
\begin{table}[htbp]
  \centering
  \caption{Caption for the table.}
  \label{tab:table1231}
  \begin{tabular}{l|c||r}
    1 & 2 & 3\\
    \hline
    a & b & c\\
  \end{tabular}
\end{table}
%
\begin{table}[htbp]
  \centering
  \caption{Caption for the table.}
  \label{tab:table1232}
  \begin{tabular}{l|c||r}
    4 & 5 & 6\\
    \hline
    d & e & f\\
  \end{tabular}
\end{table}
%
\begin{table}[htbp]
  \centering
  \caption{Caption for the table.}
  \label{tab:table1233}
  \begin{tabular}{l|c||r}
    1 & 2 & 3\\
    4 & 5 & 6\\
    \hline
    d & e & f\\
    a & b & c\\
  \end{tabular}
\end{table}
|

Se voce prefir mudar os nomes e incluir outros arquivos, voce deve
corrigir as instruções respectivas no arquivo principal
\cite{IEEEexample:phdurl}. Por exemplo, se voce criou os arquivos
\texttt{src/cap5/cap5.tex} e \texttt{src/apendice/apendice.tex}. Voce deve incluir no arquivo principal as seguintes instruções:

\noindent\verb|%TCIDATA{LaTeXparent=0,0,main.tex}

\chapter{Pictures}

Leslie Lamport~\citeauthor{lamport:latex} says "Thinking doesn't guarantee that we won't make mistakes. But not thinking guarantees that we will.". 

\section{\LaTeX and \TeX}
You can create pictures within \LaTeX {}\ using a limited set of
picture symbols. These include vector, line, oval, and others. For
more information you should refer to the \LaTeX {}\ manual
\cite{lamport:latex}.
\begin{figure}[htbp]
\centering  
\begin{picture}(100,100)(0,0)
\put(60,70){\vector(1,0){15}}
\put(65,80){$\theta^+$}
\put(60,40){\line(1,2){20}}
\put(60,40){\line(0,1){40}}
\put(60,40){\oval(40,40)[l]}
\put(60,20){\line(0,1){40}}
\put(0,40){\vector(1,0){40}}
\put(0,43){flow}
\put(20,22){\line(10,0){60}}
\put(20,22){\circle{20}}
\put(60,22){\oval(40,20)[r]}
\end{picture}
\caption{Simple picture created with \LaTeX\ .}
\end{figure}

The above picture was made with the following \LaTeX {}\ code:
\begin{verbatim}
\begin{figure}[htbp]
\centering  
\begin{picture}(100,100)(0,0)
\put(60,70){\vector(1,0){15}}
\put(65,80){$\theta^+$}
\put(60,40){\line(1,2){20}}
\put(60,40){\line(0,1){40}}
\put(60,40){\oval(40,40)[l]}
\put(60,20){\line(0,1){40}}
\put(0,40){\vector(1,0){40}}
\put(0,43){flow}
\put(20,22){\line(10,0){60}}
\put(20,22){\circle{20}}
\put(60,22){\oval(40,20)[r]}
\end{picture}
\caption{Simple picture created with \LaTeX\ .}
\end{figure}
\end{verbatim}
%
\begin{figure}[htbp]
\centering  
\setlength{\unitlength}{1cm} 
\begin{picture}(5,5)(0,0)
\linethickness{2pt} 
\put(0,0){\vector(1,2){1}}	
\put(2,3){\vector(2,3){1}}	
\put(2.2,3.2){$a_1$}	
\end{picture} 
\caption{Simple vectors created with \LaTeX\ .}
\end{figure}
%
\begin{figure}[htbp]
\centering  
\setlength{\unitlength}{1pt} 
\begin{picture}(200,200)(0,0) 
\linethickness{2pt} 
\bezier{20}(0,0)(10,30)(50,30) 
\bezier{200}(0,0)(40,0)(50,30) 
\thinlines 
\put(0,0){\circle*{1}} 
\put(0,0){\line(1,3){10}}	
\put(0,-1){\makebox(0,0)[t]{A}}	
\put(10,30){\circle*{1}} 
\put(10,31){\makebox(0,0)[b]{B}} 
\put(50,30){\circle*{1}} 
\put(50,30){\line(-1,0){40}} 
\put(50,31){\makebox(0,0)[b]{C}} 
\end{picture} 
\caption{Simple arc picture created with \LaTeX\ .}
\end{figure}
%

Another way to get pictures in your \LaTeX \ document is to use
plain \TeX {}\ commands; see The \TeX \ book \cite{knuth:tex}.

\section{TikZ}
There are other and more powerful tools to create graphic elements in LATEX. Tikz is probably the most complex and powerful tool to create graphics for \LaTeX\ . Here follows an example to show the basic features of TikZ package to draw a quite common control system block diagram.
\tikzstyle{block} = [draw, fill=blue!20, rectangle, 
    minimum height=3em, minimum width=6em]
\tikzstyle{sum} = [draw, fill=blue!20, circle, node distance=2cm]
\tikzstyle{input} = [coordinate]
\tikzstyle{output} = [coordinate]
\tikzstyle{pinstyle} = [pin edge={to-,thin,black}]
\begin{figure}[!ht]
\centering
% The block diagram code is probably more verbose than necessary
\begin{tikzpicture}[auto, node distance=3cm,>=latex']
    % We start by placing the blocks
    \node [input, name=input] {};
    \node [sum, right of=input] (sum) {};
    \node [block, right of=sum] (controller) {Controller};
    \node [block, right of=controller, pin={[pinstyle]above:Disturbances},
            node distance=4cm] (system) {System};
    % We draw an edge between the controller and system block to 
    % calculate the coordinate u. We need it to place the measurement block. 
    \draw [->] (controller) -- node[name=u] {$u$} (system);
    \node [output, right of=system] (output) {};
    \node [block, below of=u] (measurements) {Measurements};
    % Once the nodes are placed, connecting them is easy. 
    \draw [draw,->] (input) -- node {$r$} (sum);
    \draw [->] (sum) -- node {$e$} (controller);
    \draw [->] (system) -- node [name=y] {$y$}(output);
    \draw [->] (y) |- (measurements);
    \draw [->] (measurements) -| node[pos=0.99] {$-$} 
        node [near end] {$y_m$} (sum);
\end{tikzpicture}
\caption{Block diagram of a closed-loop control system.}
\end{figure}
The above picture was made with the following TikZ code:

\begin{verbatim}
\tikzstyle{block} = [draw, fill=blue!20, rectangle, 
    minimum height=3em, minimum width=6em]
\tikzstyle{sum} = [draw, fill=blue!20, circle, node distance=2cm]
\tikzstyle{input} = [coordinate]
\tikzstyle{output} = [coordinate]
\tikzstyle{pinstyle} = [pin edge={to-,thin,black}]
\begin{figure}[!ht]
\centering
% The block diagram code is probably more verbose than necessary
\begin{tikzpicture}[auto, node distance=3cm,>=latex']
    % We start by placing the blocks
    \node [input, name=input] {};
    \node [sum, right of=input] (sum) {};
    \node [block, right of=sum] (controller) {Controller};
    \node [block, right of=controller, pin={[pinstyle]above:Disturbances},
            node distance=4cm] (system) {System};
    % We draw an edge between the controller and system block to 
    % calculate the coordinate u. We need it to place the measurement block. 
    \draw [->] (controller) -- node[name=u] {$u$} (system);
    \node [output, right of=system] (output) {};
    \node [block, below of=u] (measurements) {Measurements};
    % Once the nodes are placed, connecting them is easy. 
    \draw [draw,->] (input) -- node {$r$} (sum);
    \draw [->] (sum) -- node {$e$} (controller);
    \draw [->] (system) -- node [name=y] {$y$}(output);
    \draw [->] (y) |- (measurements);
    \draw [->] (measurements) -| node[pos=0.99] {$-$} 
        node [near end] {$y_m$} (sum);
\end{tikzpicture}
\caption{Block diagram of a closed-loop control system.}
\end{figure}
\end{verbatim}
|

\noindent\verb|%% This document created by Scientific Word (R) Version 3.5

%TCIDATA{LaTeXparent=0,0,main.tex}


\chapter{Informações adicionais}

No apêndice você deve colocar as informações adicionais que
são importantes para o seu trabalho mas que, todavia, não são
essenciais e, deste modo, não devem estar na parte principal do documento.
|

Lembre que estes arquivos são do tipo secundário e não devem ter o cabeçalho padrão do \LaTeX  e nem a instrução de fim de documento, ou sejam,
\begin{verbatim}
\documentclass ... \begin{document} ... \end{document}
\end{verbatim}

Este cabeçalho so é necessário no arquivo \texttt{main.tex}. Por outro
lado, instruções do tipo
\begin{verbatim}
\chapter{nomedocapitulo} ... \section{nomedasecao}
\end{verbatim}
podem e devem ser usadas.

Observe que não é necessário acrescentar no arquivo \texttt{main.tex}
os comandos para incluir os arquivos: \texttt{src/etc/orientad.tex}, \texttt{src/etc/membrosb.tex},
\texttt{src/etc/dedicato.tex}, \texttt{src/etc/agradece.tex}, \texttt{src/etc/resumopt.tex} e \texttt{src/etc/resumoen.tex}.

Estes arquivos, devem ser, obrigatoriamente, armazenados no diretório \texttt{src/etc} e serão incluídos automaticamente no momento da compilação, como é feito no caso dos arquivos \texttt{src/etc/conclusa.tex}, \texttt{src/etc/introduc.tex}.

\section{Recomendações de redação}
Um trabalho científico pode e deve ser consultado por várias pessoas, inclusive de áreas do conhecimento diferentes daquela que você atua. Desse modo, deve-se adotar uma forma expositiva e o nível de clareza interna, fornecendo ao leitor todas as informações necessárias para a compreensão do texto, inclusive a definição dos termos empregados.

Entre algumas sugestões para a estrutura interna do texto, pode-se citar:

– Evitar períodos longos;

– Abrir parágrafos para “arejar” o texto;

– Repetir o sujeito da frase sempre que for necessário para que a compressão não seja prejudicada;

– Eliminar o excesso de pronomes e orações subordinadas;

– Suprimir as divagações, transformando-as em notas ou apêndices;

– Concentrar-se na demonstração das hipóteses levantadas;

– Verificar a facilidade de leitura do texto, solicitando que outras pessoas também o leiam;

– Evitar o emprego de reticências e pontos de exclamação;

– Usar figuras de linguagem apenas quando presumir que o leitor será capaz de compreendê-las;

– Definir um termo técnico ao induzi-lo pela primeira vez no texto;

– Ser coerente na identificação de autores e outras informações referentes às fontes documentais.

Entre as expressões que devem ser evitadas nos textos acadêmicos, sugere-se substituir:

\begin{tabular}{|L{7cm}|L{7cm}|}\hline
Expressão  &	Opções\\\hline
– a nível de, ao nível	& – em nível de, no nível\\\hline
– face a, frente a	&– ante, diante de, em vista de\\\hline
– onde (quando não exprime lugar)	& – em que, na qual, nas quais, no qual\\\hline
– sob um ponto de vista	& – de um ponto de vista\\\hline
– sob um prisma	& – por um prisma\\\hline
– como sendo	& – suprimir a expressão\\\hline
– em função de	& – em virtude de, por causa de, por,\\\hline
– a partir de (sem valor temporal)	& – com base em, tomando-se por base\\\hline
– através de (sem atravessar algo)	& – por meio de, segundo, por, mediante\\\hline
– devido a	& – em razão de, em virtude de\\\hline
– dito	& – citado, mencionado\\\hline
– como um todo	& – total, integral, completo\\\hline
– há anos atrás	& – há anos\\\hline
– por cada	& – suprimir a cacofonia\\\hline
– antes do estudo ser feito	& – antes de o estudo ser realizado\\\hline
– pelo fato destes resultados apontarem o contrário	& – pelo fato de estes resultados apontarem o contrário.\\\hline
\end{tabular}
|

\noindent\verb|%TCIDATA{LaTeXparent=0,0,main.tex}

\chapter{Citações e referências}

De modo geral é necessário fazer referência a trabalhos anteriores
que serviram de base para o desenvolvimento de seu trabalho. As
citações e a apresentação das referências no final do
documento deve seguir o padrão ABNT que pode ser implementado com o pacote

\texttt{abntex2cite}.

\noindent A lista dos trabalhos que são referenciados neste
exemplo é armazenada no arquivo \newline\texttt{src/bib/main.bib}.

\section{Referenciando}

Veja os exemplos: O programa \TeX {}\ foi idealizado e implementado
inicialmente na decada de 70 \cite{knuth:tex}. \LaTeX {}\ \cite{lamport:latex}
é uma das várias extensões do \TeX \ \cite{knuth:tex}. Há
várias implementações de \LaTeX {}\ \cite{lamport:latex} para PCs
\cite{furuta:pctex}.

\section{BibTeX}

Lembre-se de executar o BibTeX\ \cite{patashnik:bibtex} e, em seguida, executar
o \LaTeX \ \cite{lamport:latex} duas vezes consecutivas, sempre que novas
citações forem incluídas no texto.

\section{Referências}
 \lipsum[1-3]|

\noindent\verb|%TCIDATA{LaTeXparent=0,0,main.tex}

\chapter{Figuras e tabelas}

Você também pode incluir gráficos nos formatos \texttt{EPS} ou \texttt{PNG}. A escolha por um destes formatos é feita no preambulo do arquivo \texttt{main.tex}. 

A Figura~\ref{dinosaur} representa sua postura atual frente as mudanças.
Você está ciente do efeito \texttt{orloff} mas não quer pagar para
ver. \begin{figure}[htbp]
\centering \includegraphics[width=80mm]{dinossau} \caption{Voce antes de usar o LaTeX!}%
\label{dinosaur}%
\end{figure}
\begin{figure}[htbp]
\centering \includegraphics[width=80mm]{dinossau} \caption{Voce antes de usar o LaTeX!}%
\label{dinosaur1}%
\end{figure}
\begin{figure}[htbp]
\centering \includegraphics[width=80mm]{dinossau} \caption{Voce antes de usar o LaTeX!}%
\label{dinosaur2}%
\end{figure}
\begin{figure}[htbp]
\centering \includegraphics[width=80mm]{dinossau} \caption{Voce antes de usar o LaTeX!}%
\label{dinosaur3}%
\end{figure}
\begin{figure}[htbp]
\centering \includegraphics[width=80mm]{dinossau} \caption{Voce antes de usar o LaTeX!}%
\label{dinosaur4}%
\end{figure}
\begin{figure}[htbp]
\centering \includegraphics[width=80mm]{dinossau} \caption{Voce antes de usar o LaTeX!}%
\label{dinosaur5}%
\end{figure}
\begin{figure}[htbp]
\centering \includegraphics[width=80mm]{dinossau} \caption{Voce antes de usar o LaTeX!}%
\label{dinosaur6}%
\end{figure}
\begin{figure}[htbp]
\centering \includegraphics[width=80mm]{dinossau} \caption{Voce antes de usar o LaTeX!}%
\label{dinosaur7}%
\end{figure}
\begin{figure}[htbp]
\centering \includegraphics[width=80mm]{dinossau} \caption{Voce antes de usar o LaTeX!}%
\label{dinosaur8}%
\end{figure}
\begin{figure}[htbp]
\centering \includegraphics[width=80mm]{dinossau} \caption{Voce antes de usar o LaTeX!}%
\label{dinosaur9}%
\end{figure}

Já a Figura~\ref{toucan} ilustra como você se sente depois de refletir
um pouco. As pressões são muitas mas não dá para fechar
questão! Note que nao é tao estranho se se sentir assim, vários
personagens ilustres do cenário nacional também se sentem deste modo.
\begin{figure}[htbp]
\centering \includegraphics[width=80mm]{tucanobr} \caption{Voce prestes a usar o LaTeX}%
\label{toucan}%
\end{figure}
\begin{figure}[htbp]
\centering \includegraphics[width=80mm]{tucanobr} \caption{Voce prestes a usar o LaTeX}%
\label{toucan1}%
\end{figure}
\begin{figure}[htbp]
\centering \includegraphics[width=80mm]{tucanobr} \caption{Voce prestes a usar o LaTeX}%
\label{toucan2}%
\end{figure}
\begin{figure}[htbp]
\centering \includegraphics[width=80mm]{tucanobr} \caption{Voce prestes a usar o LaTeX}%
\label{toucan3}%
\end{figure}
\begin{figure}[htbp]
\centering \includegraphics[width=80mm]{tucanobr} \caption{Voce prestes a usar o LaTeX}%
\label{toucan4}%
\end{figure}
\begin{figure}[htbp]
\centering \includegraphics[width=80mm]{tucanobr} \caption{Voce prestes a usar o LaTeX}%
\label{toucan5}%
\end{figure}
\begin{figure}[htbp]
\centering \includegraphics[width=80mm]{tucanobr} \caption{Voce prestes a usar o LaTeX}%
\label{toucan6}%
\end{figure}
\begin{figure}[htbp]
\centering \includegraphics[width=80mm]{tucanobr} \caption{Voce prestes a usar o LaTeX}%
\label{toucan7}%
\end{figure}
\begin{figure}[htbp]
\centering \includegraphics[width=80mm]{tucanobr} \caption{Voce prestes a usar o LaTeX}%
\label{toucan8}%
\end{figure}
\begin{figure}[htbp]
\centering \includegraphics[width=80mm]{tucanobr} \caption{Voce prestes a usar o LaTeX}%
\label{toucan9}%
\end{figure}

\section{Formato EPS}

Para editar as suas figuras sugere-se a utilização de qualquer
programa gráfico que disponibilize um filtro para Encapsulated PostScript
\cite{Doron92e}. Uma vez que sua figura estiver pronta (observe cuidadosamente as dimensões), exporte-a como \texttt{EPS without TIFF Preview}. Lembre de alterar alterar o preambulo, de modo que as opções sejam:

\noindent\verb|\usepackage[dvips]{graphicx}|

\noindent\verb|\graphicspath{{cls/},{src/eps/}}|

\noindent\verb|\DeclareGraphicsExtensions{.eps}|\\
Se você mantiver estas escolhas, armazene seus arquivos gráficos no diretório \texttt{src/eps/}.

\section{Formato PNG}
Para editar as suas figuras sugere-se a utilização de qualquer
programa gráfico que disponibilize um filtro para Portable Network Graphics
\cite{Christensson}. Lembre de alterar alterar o preambulo, de modo que as opções sejam:

\noindent\verb|\usepackage[pdftex]{graphicx}|

\noindent\verb|\graphicspath{{cls/},{src/png/}}|

\noindent\verb|\DeclareGraphicsExtensions{.png}|\\
Se você mantiver estas escolhas, armazene seus arquivos gráficos no diretório \texttt{src/png/}.

\section{Tabelas}

Eis algumas tabelas criadas com comandos do \LaTeX . Eventualmente alguma ou
alguma delas pode lhe ser útil. Dê uma olhada e use-as como exemplos para
construir as suas!

As diversas ferramentas usadas para a preparacao de textos científicos
tem, todas, suas vantagens e desvantagens. A título de exemplo,
apresenta-se na Tabela~\ref{comparando} uma comparação entre algumas
ferramentas de edição de textos.
\begin{table}[ptb]
\centering
\begin{tabular}
[c]{|c|c|c|}\hline
Ferramenta & Curva de aprendizado & Suporte\\\hline
FrameMaker & \multicolumn{1}{|c|}{5.0} & 6.0\\\hline
Troff & \multicolumn{1}{|c|}{10.0} & 1.0\\\hline
\TeX  & \multicolumn{1}{|c|}{7.0} & 10.0\\\hline
Scientific Word/LaTeX & \multicolumn{1}{|c|}{8.0} & 10.0\\\hline
\end{tabular}
\caption{Comparando ferramentas de edição de textos}%
\label{comparando}%
\end{table}
%
\begin{table}[htbp]
\centering
\begin{tabular}
[c]{|c|c|c|}\hline
Ferramenta & Curva de aprendizado & Suporte\\\hline
FrameMaker & \multicolumn{1}{|c|}{5.0} & 6.0\\\hline
Troff & \multicolumn{1}{|c|}{10.0} & 1.0\\\hline
\TeX  & \multicolumn{1}{|c|}{7.0} & 10.0\\\hline
Scientific Word/LaTeX & \multicolumn{1}{|c|}{8.0} & 10.0\\\hline
\end{tabular}
\caption{Comparando ferramentas de edição de textos}%
\label{comparando1}%
\end{table}
%
\begin{table}[h]
\centering
\caption{Um nome qualquer}
\begin{tabular}{r|lr}
Posi{\c c}{\~a}o & Pa{\'i}s & IDH \\ % Note a separação de col. e a quebra de linhas
\hline                               % para uma linha horizontal
1 & Noruega        & .955 \\
2 & Austr{\'a}lia  & .938 \\
3 & EUA            & .937 \\
4 & Holanda        & .921 \\
5 & Alemanha       & .920            % não é preciso quebrar a última linha

\end{tabular}
\end{table}
%
\begin{table}[htbp]
  \centering
  \caption{Caption for the table.}
  \label{tab:table1}
  \begin{tabular}{l|c||r}
    1 & 2 & 3\\
    \hline
    a & b & c\\
  \end{tabular}
\end{table}
%
\begin{table}[htbp]
  \centering
  \caption{Caption for the table.}
  \label{tab:table2}
  \begin{tabular}{l|c||r}
    4 & 5 & 6\\
    \hline
    d & e & f\\
  \end{tabular}
\end{table}
%
\begin{table}[htbp]
  \centering
  \caption{Caption for the table.}
  \label{tab:table3}
  \begin{tabular}{l|c||r}
    1 & 2 & 3\\
    4 & 5 & 6\\
    \hline
    d & e & f\\
    a & b & c\\
  \end{tabular}
\end{table}
%
\begin{table}[htbp]
  \centering
  \caption{Caption for the table.}
  \label{tab:table11}
  \begin{tabular}{l|c||r}
    1 & 2 & 3\\
    \hline
    a & b & c\\
  \end{tabular}
\end{table}
%
\begin{table}[htbp]
  \centering
  \caption{Caption for the table.}
  \label{tab:table12}
  \begin{tabular}{l|c||r}
    4 & 5 & 6\\
    \hline
    d & e & f\\
  \end{tabular}
\end{table}
%
\begin{table}[htbp]
  \centering
  \caption{Caption for the table.}
  \label{tab:table13}
  \begin{tabular}{l|c||r}
    1 & 2 & 3\\
    4 & 5 & 6\\
    \hline
    d & e & f\\
    a & b & c\\
  \end{tabular}
\end{table}
%
\begin{table}[htbp]
  \centering
  \caption{Caption for the table.}
  \label{tab:table21}
  \begin{tabular}{l|c||r}
    1 & 2 & 3\\
    \hline
    a & b & c\\
  \end{tabular}
\end{table}
%
\begin{table}[htbp]
  \centering
  \caption{Caption for the table.}
  \label{tab:table22}
  \begin{tabular}{l|c||r}
    4 & 5 & 6\\
    \hline
    d & e & f\\
  \end{tabular}
\end{table}
%
\begin{table}[htbp]
  \centering
  \caption{Caption for the table.}
  \label{tab:table23}
  \begin{tabular}{l|c||r}
    1 & 2 & 3\\
    4 & 5 & 6\\
    \hline
    d & e & f\\
    a & b & c\\
  \end{tabular}
\end{table}
%
\begin{table}[htbp]
  \centering
  \caption{Caption for the table.}
  \label{tab:table31}
  \begin{tabular}{l|c||r}
    1 & 2 & 3\\
    \hline
    a & b & c\\
  \end{tabular}
\end{table}
%
\begin{table}[htbp]
  \centering
  \caption{Caption for the table.}
  \label{tab:table32}
  \begin{tabular}{l|c||r}
    4 & 5 & 6\\
    \hline
    d & e & f\\
  \end{tabular}
\end{table}
%
\begin{table}[htbp]
  \centering
  \caption{Caption for the table.}
  \label{tab:table33}
  \begin{tabular}{l|c||r}
    1 & 2 & 3\\
    4 & 5 & 6\\
    \hline
    d & e & f\\
    a & b & c\\
  \end{tabular}
\end{table}
%
\begin{table}[htbp]
  \centering
  \caption{Caption for the table.}
  \label{tab:table131}
  \begin{tabular}{l|c||r}
    1 & 2 & 3\\
    \hline
    a & b & c\\
  \end{tabular}
\end{table}
%
\begin{table}[htbp]
  \centering
  \caption{Caption for the table.}
  \label{tab:table132}
  \begin{tabular}{l|c||r}
    4 & 5 & 6\\
    \hline
    d & e & f\\
  \end{tabular}
\end{table}
%
\begin{table}[htbp]
  \centering
  \caption{Caption for the table.}
  \label{tab:table133}
  \begin{tabular}{l|c||r}
    1 & 2 & 3\\
    4 & 5 & 6\\
    \hline
    d & e & f\\
    a & b & c\\
  \end{tabular}
\end{table}
%
\begin{table}[htbp]
  \centering
  \caption{Caption for the table.}
  \label{tab:table231}
  \begin{tabular}{l|c||r}
    1 & 2 & 3\\
    \hline
    a & b & c\\
  \end{tabular}
\end{table}
%
\begin{table}[htbp]
  \centering
  \caption{Caption for the table.}
  \label{tab:table232}
  \begin{tabular}{l|c||r}
    4 & 5 & 6\\
    \hline
    d & e & f\\
  \end{tabular}
\end{table}
%
\begin{table}[htbp]
  \centering
  \caption{Caption for the table.}
  \label{tab:table233}
  \begin{tabular}{l|c||r}
    1 & 2 & 3\\
    4 & 5 & 6\\
    \hline
    d & e & f\\
    a & b & c\\
  \end{tabular}
\end{table}
%
\begin{table}[htbp]
  \centering
  \caption{Caption for the table.}
  \label{tab:table1231}
  \begin{tabular}{l|c||r}
    1 & 2 & 3\\
    \hline
    a & b & c\\
  \end{tabular}
\end{table}
%
\begin{table}[htbp]
  \centering
  \caption{Caption for the table.}
  \label{tab:table1232}
  \begin{tabular}{l|c||r}
    4 & 5 & 6\\
    \hline
    d & e & f\\
  \end{tabular}
\end{table}
%
\begin{table}[htbp]
  \centering
  \caption{Caption for the table.}
  \label{tab:table1233}
  \begin{tabular}{l|c||r}
    1 & 2 & 3\\
    4 & 5 & 6\\
    \hline
    d & e & f\\
    a & b & c\\
  \end{tabular}
\end{table}
|

Se voce prefir mudar os nomes e incluir outros arquivos, voce deve
corrigir as instruções respectivas no arquivo principal
\cite{IEEEexample:phdurl}. Por exemplo, se voce criou os arquivos
\texttt{src/cap5/cap5.tex} e \texttt{src/apendice/apendice.tex}. Voce deve incluir no arquivo principal as seguintes instruções:

\noindent\verb|%TCIDATA{LaTeXparent=0,0,main.tex}

\chapter{Pictures}

Leslie Lamport~\citeauthor{lamport:latex} says "Thinking doesn't guarantee that we won't make mistakes. But not thinking guarantees that we will.". 

\section{\LaTeX and \TeX}
You can create pictures within \LaTeX {}\ using a limited set of
picture symbols. These include vector, line, oval, and others. For
more information you should refer to the \LaTeX {}\ manual
\cite{lamport:latex}.
\begin{figure}[htbp]
\centering  
\begin{picture}(100,100)(0,0)
\put(60,70){\vector(1,0){15}}
\put(65,80){$\theta^+$}
\put(60,40){\line(1,2){20}}
\put(60,40){\line(0,1){40}}
\put(60,40){\oval(40,40)[l]}
\put(60,20){\line(0,1){40}}
\put(0,40){\vector(1,0){40}}
\put(0,43){flow}
\put(20,22){\line(10,0){60}}
\put(20,22){\circle{20}}
\put(60,22){\oval(40,20)[r]}
\end{picture}
\caption{Simple picture created with \LaTeX\ .}
\end{figure}

The above picture was made with the following \LaTeX {}\ code:
\begin{verbatim}
\begin{figure}[htbp]
\centering  
\begin{picture}(100,100)(0,0)
\put(60,70){\vector(1,0){15}}
\put(65,80){$\theta^+$}
\put(60,40){\line(1,2){20}}
\put(60,40){\line(0,1){40}}
\put(60,40){\oval(40,40)[l]}
\put(60,20){\line(0,1){40}}
\put(0,40){\vector(1,0){40}}
\put(0,43){flow}
\put(20,22){\line(10,0){60}}
\put(20,22){\circle{20}}
\put(60,22){\oval(40,20)[r]}
\end{picture}
\caption{Simple picture created with \LaTeX\ .}
\end{figure}
\end{verbatim}
%
\begin{figure}[htbp]
\centering  
\setlength{\unitlength}{1cm} 
\begin{picture}(5,5)(0,0)
\linethickness{2pt} 
\put(0,0){\vector(1,2){1}}	
\put(2,3){\vector(2,3){1}}	
\put(2.2,3.2){$a_1$}	
\end{picture} 
\caption{Simple vectors created with \LaTeX\ .}
\end{figure}
%
\begin{figure}[htbp]
\centering  
\setlength{\unitlength}{1pt} 
\begin{picture}(200,200)(0,0) 
\linethickness{2pt} 
\bezier{20}(0,0)(10,30)(50,30) 
\bezier{200}(0,0)(40,0)(50,30) 
\thinlines 
\put(0,0){\circle*{1}} 
\put(0,0){\line(1,3){10}}	
\put(0,-1){\makebox(0,0)[t]{A}}	
\put(10,30){\circle*{1}} 
\put(10,31){\makebox(0,0)[b]{B}} 
\put(50,30){\circle*{1}} 
\put(50,30){\line(-1,0){40}} 
\put(50,31){\makebox(0,0)[b]{C}} 
\end{picture} 
\caption{Simple arc picture created with \LaTeX\ .}
\end{figure}
%

Another way to get pictures in your \LaTeX \ document is to use
plain \TeX {}\ commands; see The \TeX \ book \cite{knuth:tex}.

\section{TikZ}
There are other and more powerful tools to create graphic elements in LATEX. Tikz is probably the most complex and powerful tool to create graphics for \LaTeX\ . Here follows an example to show the basic features of TikZ package to draw a quite common control system block diagram.
\tikzstyle{block} = [draw, fill=blue!20, rectangle, 
    minimum height=3em, minimum width=6em]
\tikzstyle{sum} = [draw, fill=blue!20, circle, node distance=2cm]
\tikzstyle{input} = [coordinate]
\tikzstyle{output} = [coordinate]
\tikzstyle{pinstyle} = [pin edge={to-,thin,black}]
\begin{figure}[!ht]
\centering
% The block diagram code is probably more verbose than necessary
\begin{tikzpicture}[auto, node distance=3cm,>=latex']
    % We start by placing the blocks
    \node [input, name=input] {};
    \node [sum, right of=input] (sum) {};
    \node [block, right of=sum] (controller) {Controller};
    \node [block, right of=controller, pin={[pinstyle]above:Disturbances},
            node distance=4cm] (system) {System};
    % We draw an edge between the controller and system block to 
    % calculate the coordinate u. We need it to place the measurement block. 
    \draw [->] (controller) -- node[name=u] {$u$} (system);
    \node [output, right of=system] (output) {};
    \node [block, below of=u] (measurements) {Measurements};
    % Once the nodes are placed, connecting them is easy. 
    \draw [draw,->] (input) -- node {$r$} (sum);
    \draw [->] (sum) -- node {$e$} (controller);
    \draw [->] (system) -- node [name=y] {$y$}(output);
    \draw [->] (y) |- (measurements);
    \draw [->] (measurements) -| node[pos=0.99] {$-$} 
        node [near end] {$y_m$} (sum);
\end{tikzpicture}
\caption{Block diagram of a closed-loop control system.}
\end{figure}
The above picture was made with the following TikZ code:

\begin{verbatim}
\tikzstyle{block} = [draw, fill=blue!20, rectangle, 
    minimum height=3em, minimum width=6em]
\tikzstyle{sum} = [draw, fill=blue!20, circle, node distance=2cm]
\tikzstyle{input} = [coordinate]
\tikzstyle{output} = [coordinate]
\tikzstyle{pinstyle} = [pin edge={to-,thin,black}]
\begin{figure}[!ht]
\centering
% The block diagram code is probably more verbose than necessary
\begin{tikzpicture}[auto, node distance=3cm,>=latex']
    % We start by placing the blocks
    \node [input, name=input] {};
    \node [sum, right of=input] (sum) {};
    \node [block, right of=sum] (controller) {Controller};
    \node [block, right of=controller, pin={[pinstyle]above:Disturbances},
            node distance=4cm] (system) {System};
    % We draw an edge between the controller and system block to 
    % calculate the coordinate u. We need it to place the measurement block. 
    \draw [->] (controller) -- node[name=u] {$u$} (system);
    \node [output, right of=system] (output) {};
    \node [block, below of=u] (measurements) {Measurements};
    % Once the nodes are placed, connecting them is easy. 
    \draw [draw,->] (input) -- node {$r$} (sum);
    \draw [->] (sum) -- node {$e$} (controller);
    \draw [->] (system) -- node [name=y] {$y$}(output);
    \draw [->] (y) |- (measurements);
    \draw [->] (measurements) -| node[pos=0.99] {$-$} 
        node [near end] {$y_m$} (sum);
\end{tikzpicture}
\caption{Block diagram of a closed-loop control system.}
\end{figure}
\end{verbatim}
|

\noindent\verb|%% This document created by Scientific Word (R) Version 3.5

%TCIDATA{LaTeXparent=0,0,main.tex}


\chapter{Informações adicionais}

No apêndice você deve colocar as informações adicionais que
são importantes para o seu trabalho mas que, todavia, não são
essenciais e, deste modo, não devem estar na parte principal do documento.
|

Lembre que estes arquivos são do tipo secundário e não devem ter o cabeçalho padrão do \LaTeX  e nem a instrução de fim de documento, ou sejam,
\begin{verbatim}
\documentclass ... \begin{document} ... \end{document}
\end{verbatim}

Este cabeçalho so é necessário no arquivo \texttt{main.tex}. Por outro
lado, instruções do tipo
\begin{verbatim}
\chapter{nomedocapitulo} ... \section{nomedasecao}
\end{verbatim}
podem e devem ser usadas.

Observe que não é necessário acrescentar no arquivo \texttt{main.tex}
os comandos para incluir os arquivos: \texttt{src/etc/orientad.tex}, \texttt{src/etc/membrosb.tex},
\texttt{src/etc/dedicato.tex}, \texttt{src/etc/agradece.tex}, \texttt{src/etc/resumopt.tex} e \texttt{src/etc/resumoen.tex}.

Estes arquivos, devem ser, obrigatoriamente, armazenados no diretório \texttt{src/etc} e serão incluídos automaticamente no momento da compilação, como é feito no caso dos arquivos \texttt{src/etc/conclusa.tex}, \texttt{src/etc/introduc.tex}.

\section{Recomendações de redação}
Um trabalho científico pode e deve ser consultado por várias pessoas, inclusive de áreas do conhecimento diferentes daquela que você atua. Desse modo, deve-se adotar uma forma expositiva e o nível de clareza interna, fornecendo ao leitor todas as informações necessárias para a compreensão do texto, inclusive a definição dos termos empregados.

Entre algumas sugestões para a estrutura interna do texto, pode-se citar:

– Evitar períodos longos;

– Abrir parágrafos para “arejar” o texto;

– Repetir o sujeito da frase sempre que for necessário para que a compressão não seja prejudicada;

– Eliminar o excesso de pronomes e orações subordinadas;

– Suprimir as divagações, transformando-as em notas ou apêndices;

– Concentrar-se na demonstração das hipóteses levantadas;

– Verificar a facilidade de leitura do texto, solicitando que outras pessoas também o leiam;

– Evitar o emprego de reticências e pontos de exclamação;

– Usar figuras de linguagem apenas quando presumir que o leitor será capaz de compreendê-las;

– Definir um termo técnico ao induzi-lo pela primeira vez no texto;

– Ser coerente na identificação de autores e outras informações referentes às fontes documentais.

Entre as expressões que devem ser evitadas nos textos acadêmicos, sugere-se substituir:

\begin{tabular}{|L{7cm}|L{7cm}|}\hline
Expressão  &	Opções\\\hline
– a nível de, ao nível	& – em nível de, no nível\\\hline
– face a, frente a	&– ante, diante de, em vista de\\\hline
– onde (quando não exprime lugar)	& – em que, na qual, nas quais, no qual\\\hline
– sob um ponto de vista	& – de um ponto de vista\\\hline
– sob um prisma	& – por um prisma\\\hline
– como sendo	& – suprimir a expressão\\\hline
– em função de	& – em virtude de, por causa de, por,\\\hline
– a partir de (sem valor temporal)	& – com base em, tomando-se por base\\\hline
– através de (sem atravessar algo)	& – por meio de, segundo, por, mediante\\\hline
– devido a	& – em razão de, em virtude de\\\hline
– dito	& – citado, mencionado\\\hline
– como um todo	& – total, integral, completo\\\hline
– há anos atrás	& – há anos\\\hline
– por cada	& – suprimir a cacofonia\\\hline
– antes do estudo ser feito	& – antes de o estudo ser realizado\\\hline
– pelo fato destes resultados apontarem o contrário	& – pelo fato de estes resultados apontarem o contrário.\\\hline
\end{tabular}
|

\noindent\verb|%TCIDATA{LaTeXparent=0,0,main.tex}

\chapter{Citações e referências}

De modo geral é necessário fazer referência a trabalhos anteriores
que serviram de base para o desenvolvimento de seu trabalho. As
citações e a apresentação das referências no final do
documento deve seguir o padrão ABNT que pode ser implementado com o pacote

\texttt{abntex2cite}.

\noindent A lista dos trabalhos que são referenciados neste
exemplo é armazenada no arquivo \newline\texttt{src/bib/main.bib}.

\section{Referenciando}

Veja os exemplos: O programa \TeX {}\ foi idealizado e implementado
inicialmente na decada de 70 \cite{knuth:tex}. \LaTeX {}\ \cite{lamport:latex}
é uma das várias extensões do \TeX \ \cite{knuth:tex}. Há
várias implementações de \LaTeX {}\ \cite{lamport:latex} para PCs
\cite{furuta:pctex}.

\section{BibTeX}

Lembre-se de executar o BibTeX\ \cite{patashnik:bibtex} e, em seguida, executar
o \LaTeX \ \cite{lamport:latex} duas vezes consecutivas, sempre que novas
citações forem incluídas no texto.

\section{Referências}
 \lipsum[1-3]|

\noindent\verb|%TCIDATA{LaTeXparent=0,0,main.tex}

\chapter{Figuras e tabelas}

Você também pode incluir gráficos nos formatos \texttt{EPS} ou \texttt{PNG}. A escolha por um destes formatos é feita no preambulo do arquivo \texttt{main.tex}. 

A Figura~\ref{dinosaur} representa sua postura atual frente as mudanças.
Você está ciente do efeito \texttt{orloff} mas não quer pagar para
ver. \begin{figure}[htbp]
\centering \includegraphics[width=80mm]{dinossau} \caption{Voce antes de usar o LaTeX!}%
\label{dinosaur}%
\end{figure}
\begin{figure}[htbp]
\centering \includegraphics[width=80mm]{dinossau} \caption{Voce antes de usar o LaTeX!}%
\label{dinosaur1}%
\end{figure}
\begin{figure}[htbp]
\centering \includegraphics[width=80mm]{dinossau} \caption{Voce antes de usar o LaTeX!}%
\label{dinosaur2}%
\end{figure}
\begin{figure}[htbp]
\centering \includegraphics[width=80mm]{dinossau} \caption{Voce antes de usar o LaTeX!}%
\label{dinosaur3}%
\end{figure}
\begin{figure}[htbp]
\centering \includegraphics[width=80mm]{dinossau} \caption{Voce antes de usar o LaTeX!}%
\label{dinosaur4}%
\end{figure}
\begin{figure}[htbp]
\centering \includegraphics[width=80mm]{dinossau} \caption{Voce antes de usar o LaTeX!}%
\label{dinosaur5}%
\end{figure}
\begin{figure}[htbp]
\centering \includegraphics[width=80mm]{dinossau} \caption{Voce antes de usar o LaTeX!}%
\label{dinosaur6}%
\end{figure}
\begin{figure}[htbp]
\centering \includegraphics[width=80mm]{dinossau} \caption{Voce antes de usar o LaTeX!}%
\label{dinosaur7}%
\end{figure}
\begin{figure}[htbp]
\centering \includegraphics[width=80mm]{dinossau} \caption{Voce antes de usar o LaTeX!}%
\label{dinosaur8}%
\end{figure}
\begin{figure}[htbp]
\centering \includegraphics[width=80mm]{dinossau} \caption{Voce antes de usar o LaTeX!}%
\label{dinosaur9}%
\end{figure}

Já a Figura~\ref{toucan} ilustra como você se sente depois de refletir
um pouco. As pressões são muitas mas não dá para fechar
questão! Note que nao é tao estranho se se sentir assim, vários
personagens ilustres do cenário nacional também se sentem deste modo.
\begin{figure}[htbp]
\centering \includegraphics[width=80mm]{tucanobr} \caption{Voce prestes a usar o LaTeX}%
\label{toucan}%
\end{figure}
\begin{figure}[htbp]
\centering \includegraphics[width=80mm]{tucanobr} \caption{Voce prestes a usar o LaTeX}%
\label{toucan1}%
\end{figure}
\begin{figure}[htbp]
\centering \includegraphics[width=80mm]{tucanobr} \caption{Voce prestes a usar o LaTeX}%
\label{toucan2}%
\end{figure}
\begin{figure}[htbp]
\centering \includegraphics[width=80mm]{tucanobr} \caption{Voce prestes a usar o LaTeX}%
\label{toucan3}%
\end{figure}
\begin{figure}[htbp]
\centering \includegraphics[width=80mm]{tucanobr} \caption{Voce prestes a usar o LaTeX}%
\label{toucan4}%
\end{figure}
\begin{figure}[htbp]
\centering \includegraphics[width=80mm]{tucanobr} \caption{Voce prestes a usar o LaTeX}%
\label{toucan5}%
\end{figure}
\begin{figure}[htbp]
\centering \includegraphics[width=80mm]{tucanobr} \caption{Voce prestes a usar o LaTeX}%
\label{toucan6}%
\end{figure}
\begin{figure}[htbp]
\centering \includegraphics[width=80mm]{tucanobr} \caption{Voce prestes a usar o LaTeX}%
\label{toucan7}%
\end{figure}
\begin{figure}[htbp]
\centering \includegraphics[width=80mm]{tucanobr} \caption{Voce prestes a usar o LaTeX}%
\label{toucan8}%
\end{figure}
\begin{figure}[htbp]
\centering \includegraphics[width=80mm]{tucanobr} \caption{Voce prestes a usar o LaTeX}%
\label{toucan9}%
\end{figure}

\section{Formato EPS}

Para editar as suas figuras sugere-se a utilização de qualquer
programa gráfico que disponibilize um filtro para Encapsulated PostScript
\cite{Doron92e}. Uma vez que sua figura estiver pronta (observe cuidadosamente as dimensões), exporte-a como \texttt{EPS without TIFF Preview}. Lembre de alterar alterar o preambulo, de modo que as opções sejam:

\noindent\verb|\usepackage[dvips]{graphicx}|

\noindent\verb|\graphicspath{{cls/},{src/eps/}}|

\noindent\verb|\DeclareGraphicsExtensions{.eps}|\\
Se você mantiver estas escolhas, armazene seus arquivos gráficos no diretório \texttt{src/eps/}.

\section{Formato PNG}
Para editar as suas figuras sugere-se a utilização de qualquer
programa gráfico que disponibilize um filtro para Portable Network Graphics
\cite{Christensson}. Lembre de alterar alterar o preambulo, de modo que as opções sejam:

\noindent\verb|\usepackage[pdftex]{graphicx}|

\noindent\verb|\graphicspath{{cls/},{src/png/}}|

\noindent\verb|\DeclareGraphicsExtensions{.png}|\\
Se você mantiver estas escolhas, armazene seus arquivos gráficos no diretório \texttt{src/png/}.

\section{Tabelas}

Eis algumas tabelas criadas com comandos do \LaTeX . Eventualmente alguma ou
alguma delas pode lhe ser útil. Dê uma olhada e use-as como exemplos para
construir as suas!

As diversas ferramentas usadas para a preparacao de textos científicos
tem, todas, suas vantagens e desvantagens. A título de exemplo,
apresenta-se na Tabela~\ref{comparando} uma comparação entre algumas
ferramentas de edição de textos.
\begin{table}[ptb]
\centering
\begin{tabular}
[c]{|c|c|c|}\hline
Ferramenta & Curva de aprendizado & Suporte\\\hline
FrameMaker & \multicolumn{1}{|c|}{5.0} & 6.0\\\hline
Troff & \multicolumn{1}{|c|}{10.0} & 1.0\\\hline
\TeX  & \multicolumn{1}{|c|}{7.0} & 10.0\\\hline
Scientific Word/LaTeX & \multicolumn{1}{|c|}{8.0} & 10.0\\\hline
\end{tabular}
\caption{Comparando ferramentas de edição de textos}%
\label{comparando}%
\end{table}
%
\begin{table}[htbp]
\centering
\begin{tabular}
[c]{|c|c|c|}\hline
Ferramenta & Curva de aprendizado & Suporte\\\hline
FrameMaker & \multicolumn{1}{|c|}{5.0} & 6.0\\\hline
Troff & \multicolumn{1}{|c|}{10.0} & 1.0\\\hline
\TeX  & \multicolumn{1}{|c|}{7.0} & 10.0\\\hline
Scientific Word/LaTeX & \multicolumn{1}{|c|}{8.0} & 10.0\\\hline
\end{tabular}
\caption{Comparando ferramentas de edição de textos}%
\label{comparando1}%
\end{table}
%
\begin{table}[h]
\centering
\caption{Um nome qualquer}
\begin{tabular}{r|lr}
Posi{\c c}{\~a}o & Pa{\'i}s & IDH \\ % Note a separação de col. e a quebra de linhas
\hline                               % para uma linha horizontal
1 & Noruega        & .955 \\
2 & Austr{\'a}lia  & .938 \\
3 & EUA            & .937 \\
4 & Holanda        & .921 \\
5 & Alemanha       & .920            % não é preciso quebrar a última linha

\end{tabular}
\end{table}
%
\begin{table}[htbp]
  \centering
  \caption{Caption for the table.}
  \label{tab:table1}
  \begin{tabular}{l|c||r}
    1 & 2 & 3\\
    \hline
    a & b & c\\
  \end{tabular}
\end{table}
%
\begin{table}[htbp]
  \centering
  \caption{Caption for the table.}
  \label{tab:table2}
  \begin{tabular}{l|c||r}
    4 & 5 & 6\\
    \hline
    d & e & f\\
  \end{tabular}
\end{table}
%
\begin{table}[htbp]
  \centering
  \caption{Caption for the table.}
  \label{tab:table3}
  \begin{tabular}{l|c||r}
    1 & 2 & 3\\
    4 & 5 & 6\\
    \hline
    d & e & f\\
    a & b & c\\
  \end{tabular}
\end{table}
%
\begin{table}[htbp]
  \centering
  \caption{Caption for the table.}
  \label{tab:table11}
  \begin{tabular}{l|c||r}
    1 & 2 & 3\\
    \hline
    a & b & c\\
  \end{tabular}
\end{table}
%
\begin{table}[htbp]
  \centering
  \caption{Caption for the table.}
  \label{tab:table12}
  \begin{tabular}{l|c||r}
    4 & 5 & 6\\
    \hline
    d & e & f\\
  \end{tabular}
\end{table}
%
\begin{table}[htbp]
  \centering
  \caption{Caption for the table.}
  \label{tab:table13}
  \begin{tabular}{l|c||r}
    1 & 2 & 3\\
    4 & 5 & 6\\
    \hline
    d & e & f\\
    a & b & c\\
  \end{tabular}
\end{table}
%
\begin{table}[htbp]
  \centering
  \caption{Caption for the table.}
  \label{tab:table21}
  \begin{tabular}{l|c||r}
    1 & 2 & 3\\
    \hline
    a & b & c\\
  \end{tabular}
\end{table}
%
\begin{table}[htbp]
  \centering
  \caption{Caption for the table.}
  \label{tab:table22}
  \begin{tabular}{l|c||r}
    4 & 5 & 6\\
    \hline
    d & e & f\\
  \end{tabular}
\end{table}
%
\begin{table}[htbp]
  \centering
  \caption{Caption for the table.}
  \label{tab:table23}
  \begin{tabular}{l|c||r}
    1 & 2 & 3\\
    4 & 5 & 6\\
    \hline
    d & e & f\\
    a & b & c\\
  \end{tabular}
\end{table}
%
\begin{table}[htbp]
  \centering
  \caption{Caption for the table.}
  \label{tab:table31}
  \begin{tabular}{l|c||r}
    1 & 2 & 3\\
    \hline
    a & b & c\\
  \end{tabular}
\end{table}
%
\begin{table}[htbp]
  \centering
  \caption{Caption for the table.}
  \label{tab:table32}
  \begin{tabular}{l|c||r}
    4 & 5 & 6\\
    \hline
    d & e & f\\
  \end{tabular}
\end{table}
%
\begin{table}[htbp]
  \centering
  \caption{Caption for the table.}
  \label{tab:table33}
  \begin{tabular}{l|c||r}
    1 & 2 & 3\\
    4 & 5 & 6\\
    \hline
    d & e & f\\
    a & b & c\\
  \end{tabular}
\end{table}
%
\begin{table}[htbp]
  \centering
  \caption{Caption for the table.}
  \label{tab:table131}
  \begin{tabular}{l|c||r}
    1 & 2 & 3\\
    \hline
    a & b & c\\
  \end{tabular}
\end{table}
%
\begin{table}[htbp]
  \centering
  \caption{Caption for the table.}
  \label{tab:table132}
  \begin{tabular}{l|c||r}
    4 & 5 & 6\\
    \hline
    d & e & f\\
  \end{tabular}
\end{table}
%
\begin{table}[htbp]
  \centering
  \caption{Caption for the table.}
  \label{tab:table133}
  \begin{tabular}{l|c||r}
    1 & 2 & 3\\
    4 & 5 & 6\\
    \hline
    d & e & f\\
    a & b & c\\
  \end{tabular}
\end{table}
%
\begin{table}[htbp]
  \centering
  \caption{Caption for the table.}
  \label{tab:table231}
  \begin{tabular}{l|c||r}
    1 & 2 & 3\\
    \hline
    a & b & c\\
  \end{tabular}
\end{table}
%
\begin{table}[htbp]
  \centering
  \caption{Caption for the table.}
  \label{tab:table232}
  \begin{tabular}{l|c||r}
    4 & 5 & 6\\
    \hline
    d & e & f\\
  \end{tabular}
\end{table}
%
\begin{table}[htbp]
  \centering
  \caption{Caption for the table.}
  \label{tab:table233}
  \begin{tabular}{l|c||r}
    1 & 2 & 3\\
    4 & 5 & 6\\
    \hline
    d & e & f\\
    a & b & c\\
  \end{tabular}
\end{table}
%
\begin{table}[htbp]
  \centering
  \caption{Caption for the table.}
  \label{tab:table1231}
  \begin{tabular}{l|c||r}
    1 & 2 & 3\\
    \hline
    a & b & c\\
  \end{tabular}
\end{table}
%
\begin{table}[htbp]
  \centering
  \caption{Caption for the table.}
  \label{tab:table1232}
  \begin{tabular}{l|c||r}
    4 & 5 & 6\\
    \hline
    d & e & f\\
  \end{tabular}
\end{table}
%
\begin{table}[htbp]
  \centering
  \caption{Caption for the table.}
  \label{tab:table1233}
  \begin{tabular}{l|c||r}
    1 & 2 & 3\\
    4 & 5 & 6\\
    \hline
    d & e & f\\
    a & b & c\\
  \end{tabular}
\end{table}
|

Se voce prefir mudar os nomes e incluir outros arquivos, voce deve
corrigir as instruções respectivas no arquivo principal
\cite{IEEEexample:phdurl}. Por exemplo, se voce criou os arquivos
\texttt{src/cap5/cap5.tex} e \texttt{src/apendice/apendice.tex}. Voce deve incluir no arquivo principal as seguintes instruções:

\noindent\verb|%TCIDATA{LaTeXparent=0,0,main.tex}

\chapter{Pictures}

Leslie Lamport~\citeauthor{lamport:latex} says "Thinking doesn't guarantee that we won't make mistakes. But not thinking guarantees that we will.". 

\section{\LaTeX and \TeX}
You can create pictures within \LaTeX {}\ using a limited set of
picture symbols. These include vector, line, oval, and others. For
more information you should refer to the \LaTeX {}\ manual
\cite{lamport:latex}.
\begin{figure}[htbp]
\centering  
\begin{picture}(100,100)(0,0)
\put(60,70){\vector(1,0){15}}
\put(65,80){$\theta^+$}
\put(60,40){\line(1,2){20}}
\put(60,40){\line(0,1){40}}
\put(60,40){\oval(40,40)[l]}
\put(60,20){\line(0,1){40}}
\put(0,40){\vector(1,0){40}}
\put(0,43){flow}
\put(20,22){\line(10,0){60}}
\put(20,22){\circle{20}}
\put(60,22){\oval(40,20)[r]}
\end{picture}
\caption{Simple picture created with \LaTeX\ .}
\end{figure}

The above picture was made with the following \LaTeX {}\ code:
\begin{verbatim}
\begin{figure}[htbp]
\centering  
\begin{picture}(100,100)(0,0)
\put(60,70){\vector(1,0){15}}
\put(65,80){$\theta^+$}
\put(60,40){\line(1,2){20}}
\put(60,40){\line(0,1){40}}
\put(60,40){\oval(40,40)[l]}
\put(60,20){\line(0,1){40}}
\put(0,40){\vector(1,0){40}}
\put(0,43){flow}
\put(20,22){\line(10,0){60}}
\put(20,22){\circle{20}}
\put(60,22){\oval(40,20)[r]}
\end{picture}
\caption{Simple picture created with \LaTeX\ .}
\end{figure}
\end{verbatim}
%
\begin{figure}[htbp]
\centering  
\setlength{\unitlength}{1cm} 
\begin{picture}(5,5)(0,0)
\linethickness{2pt} 
\put(0,0){\vector(1,2){1}}	
\put(2,3){\vector(2,3){1}}	
\put(2.2,3.2){$a_1$}	
\end{picture} 
\caption{Simple vectors created with \LaTeX\ .}
\end{figure}
%
\begin{figure}[htbp]
\centering  
\setlength{\unitlength}{1pt} 
\begin{picture}(200,200)(0,0) 
\linethickness{2pt} 
\bezier{20}(0,0)(10,30)(50,30) 
\bezier{200}(0,0)(40,0)(50,30) 
\thinlines 
\put(0,0){\circle*{1}} 
\put(0,0){\line(1,3){10}}	
\put(0,-1){\makebox(0,0)[t]{A}}	
\put(10,30){\circle*{1}} 
\put(10,31){\makebox(0,0)[b]{B}} 
\put(50,30){\circle*{1}} 
\put(50,30){\line(-1,0){40}} 
\put(50,31){\makebox(0,0)[b]{C}} 
\end{picture} 
\caption{Simple arc picture created with \LaTeX\ .}
\end{figure}
%

Another way to get pictures in your \LaTeX \ document is to use
plain \TeX {}\ commands; see The \TeX \ book \cite{knuth:tex}.

\section{TikZ}
There are other and more powerful tools to create graphic elements in LATEX. Tikz is probably the most complex and powerful tool to create graphics for \LaTeX\ . Here follows an example to show the basic features of TikZ package to draw a quite common control system block diagram.
\tikzstyle{block} = [draw, fill=blue!20, rectangle, 
    minimum height=3em, minimum width=6em]
\tikzstyle{sum} = [draw, fill=blue!20, circle, node distance=2cm]
\tikzstyle{input} = [coordinate]
\tikzstyle{output} = [coordinate]
\tikzstyle{pinstyle} = [pin edge={to-,thin,black}]
\begin{figure}[!ht]
\centering
% The block diagram code is probably more verbose than necessary
\begin{tikzpicture}[auto, node distance=3cm,>=latex']
    % We start by placing the blocks
    \node [input, name=input] {};
    \node [sum, right of=input] (sum) {};
    \node [block, right of=sum] (controller) {Controller};
    \node [block, right of=controller, pin={[pinstyle]above:Disturbances},
            node distance=4cm] (system) {System};
    % We draw an edge between the controller and system block to 
    % calculate the coordinate u. We need it to place the measurement block. 
    \draw [->] (controller) -- node[name=u] {$u$} (system);
    \node [output, right of=system] (output) {};
    \node [block, below of=u] (measurements) {Measurements};
    % Once the nodes are placed, connecting them is easy. 
    \draw [draw,->] (input) -- node {$r$} (sum);
    \draw [->] (sum) -- node {$e$} (controller);
    \draw [->] (system) -- node [name=y] {$y$}(output);
    \draw [->] (y) |- (measurements);
    \draw [->] (measurements) -| node[pos=0.99] {$-$} 
        node [near end] {$y_m$} (sum);
\end{tikzpicture}
\caption{Block diagram of a closed-loop control system.}
\end{figure}
The above picture was made with the following TikZ code:

\begin{verbatim}
\tikzstyle{block} = [draw, fill=blue!20, rectangle, 
    minimum height=3em, minimum width=6em]
\tikzstyle{sum} = [draw, fill=blue!20, circle, node distance=2cm]
\tikzstyle{input} = [coordinate]
\tikzstyle{output} = [coordinate]
\tikzstyle{pinstyle} = [pin edge={to-,thin,black}]
\begin{figure}[!ht]
\centering
% The block diagram code is probably more verbose than necessary
\begin{tikzpicture}[auto, node distance=3cm,>=latex']
    % We start by placing the blocks
    \node [input, name=input] {};
    \node [sum, right of=input] (sum) {};
    \node [block, right of=sum] (controller) {Controller};
    \node [block, right of=controller, pin={[pinstyle]above:Disturbances},
            node distance=4cm] (system) {System};
    % We draw an edge between the controller and system block to 
    % calculate the coordinate u. We need it to place the measurement block. 
    \draw [->] (controller) -- node[name=u] {$u$} (system);
    \node [output, right of=system] (output) {};
    \node [block, below of=u] (measurements) {Measurements};
    % Once the nodes are placed, connecting them is easy. 
    \draw [draw,->] (input) -- node {$r$} (sum);
    \draw [->] (sum) -- node {$e$} (controller);
    \draw [->] (system) -- node [name=y] {$y$}(output);
    \draw [->] (y) |- (measurements);
    \draw [->] (measurements) -| node[pos=0.99] {$-$} 
        node [near end] {$y_m$} (sum);
\end{tikzpicture}
\caption{Block diagram of a closed-loop control system.}
\end{figure}
\end{verbatim}
|

\noindent\verb|%% This document created by Scientific Word (R) Version 3.5

%TCIDATA{LaTeXparent=0,0,main.tex}


\chapter{Informações adicionais}

No apêndice você deve colocar as informações adicionais que
são importantes para o seu trabalho mas que, todavia, não são
essenciais e, deste modo, não devem estar na parte principal do documento.
|

Lembre que estes arquivos são do tipo secundário e não devem ter o cabeçalho padrão do \LaTeX  e nem a instrução de fim de documento, ou sejam,
\begin{verbatim}
\documentclass ... \begin{document} ... \end{document}
\end{verbatim}

Este cabeçalho so é necessário no arquivo \texttt{main.tex}. Por outro
lado, instruções do tipo
\begin{verbatim}
\chapter{nomedocapitulo} ... \section{nomedasecao}
\end{verbatim}
podem e devem ser usadas.

Observe que não é necessário acrescentar no arquivo \texttt{main.tex}
os comandos para incluir os arquivos: \texttt{src/etc/orientad.tex}, \texttt{src/etc/membrosb.tex},
\texttt{src/etc/dedicato.tex}, \texttt{src/etc/agradece.tex}, \texttt{src/etc/resumopt.tex} e \texttt{src/etc/resumoen.tex}.

Estes arquivos, devem ser, obrigatoriamente, armazenados no diretório \texttt{src/etc} e serão incluídos automaticamente no momento da compilação, como é feito no caso dos arquivos \texttt{src/etc/conclusa.tex}, \texttt{src/etc/introduc.tex}.

\section{Recomendações de redação}
Um trabalho científico pode e deve ser consultado por várias pessoas, inclusive de áreas do conhecimento diferentes daquela que você atua. Desse modo, deve-se adotar uma forma expositiva e o nível de clareza interna, fornecendo ao leitor todas as informações necessárias para a compreensão do texto, inclusive a definição dos termos empregados.

Entre algumas sugestões para a estrutura interna do texto, pode-se citar:

– Evitar períodos longos;

– Abrir parágrafos para “arejar” o texto;

– Repetir o sujeito da frase sempre que for necessário para que a compressão não seja prejudicada;

– Eliminar o excesso de pronomes e orações subordinadas;

– Suprimir as divagações, transformando-as em notas ou apêndices;

– Concentrar-se na demonstração das hipóteses levantadas;

– Verificar a facilidade de leitura do texto, solicitando que outras pessoas também o leiam;

– Evitar o emprego de reticências e pontos de exclamação;

– Usar figuras de linguagem apenas quando presumir que o leitor será capaz de compreendê-las;

– Definir um termo técnico ao induzi-lo pela primeira vez no texto;

– Ser coerente na identificação de autores e outras informações referentes às fontes documentais.

Entre as expressões que devem ser evitadas nos textos acadêmicos, sugere-se substituir:

\begin{tabular}{|L{7cm}|L{7cm}|}\hline
Expressão  &	Opções\\\hline
– a nível de, ao nível	& – em nível de, no nível\\\hline
– face a, frente a	&– ante, diante de, em vista de\\\hline
– onde (quando não exprime lugar)	& – em que, na qual, nas quais, no qual\\\hline
– sob um ponto de vista	& – de um ponto de vista\\\hline
– sob um prisma	& – por um prisma\\\hline
– como sendo	& – suprimir a expressão\\\hline
– em função de	& – em virtude de, por causa de, por,\\\hline
– a partir de (sem valor temporal)	& – com base em, tomando-se por base\\\hline
– através de (sem atravessar algo)	& – por meio de, segundo, por, mediante\\\hline
– devido a	& – em razão de, em virtude de\\\hline
– dito	& – citado, mencionado\\\hline
– como um todo	& – total, integral, completo\\\hline
– há anos atrás	& – há anos\\\hline
– por cada	& – suprimir a cacofonia\\\hline
– antes do estudo ser feito	& – antes de o estudo ser realizado\\\hline
– pelo fato destes resultados apontarem o contrário	& – pelo fato de estes resultados apontarem o contrário.\\\hline
\end{tabular}
